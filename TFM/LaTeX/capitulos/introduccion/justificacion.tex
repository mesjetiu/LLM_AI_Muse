\section{Justificación}

% \defaultFontEpigraph{All You Need Is Beyond a Good Init}{\cite{xieAllYouNeed2017}}

% \defaultFontEpigraph{All You Need Is Beyond a Good Init}{Xie, Xiong, y Pu (2017)}

La evolución de la \gls{ia}, específicamente en el dominio del aprendizaje profundo o \gls{dl}, ha llevado al desarrollo de {modelos de lenguaje a gran escala} o \glsdisp{llm}{\emph{large language models} (LLM)}. Estos modelos, originalmente diseñados para emular el lenguaje humano, y conocidos por el gran público a través de aplicaciones tipo {chatbot}\footnote{Entre las aplicaciones de {chatbot} más conocidas están {ChatGPT} \citep{IntroducingChatGPT}, {Claude} \citep{IntroducingClaude}, {Bard} \citep{BardChatbot2024} o {Microsoft Copilot} \citep{mehdiAnnouncingMicrosoftCopilot2023}, aunque la lista cada vez es más larga e incluye modelos de lenguaje de código abierto como {LLaMa} \citep{touvronLLaMAOpenEfficient2023}, {Mistral} \citep{jiangMistral7B2023} o {Falcon} \citep{almazroueiFalconSeriesOpen2023}, entre otros.} han demostrado capacidades emergentes, como la creatividad o el razonamiento, en todas las áreas del conocimiento y acción humanos. Su éxito se extiende a sectores diversos como la literatura, el arte visual, la investigación, el marketing, la educación o la música. 

Al mismo tiempo, los \gls{llm} han probado ser especialmente eficaces en la generación de código de programación, entre los que podemos incluir lenguajes relacionados con la creación musical o sonora, siempre que el modelo haya contado para su entrenamiento con suficiente material escrito en el lenguaje en cuestión. Es en esta intersección entre la generación de código y la creación musical donde se sitúa el objeto de estudio de este trabajo, elemento que, hasta donde alcanza nuestro conocimiento, no ha sido objeto de estudio en profundidad en la literatura científica.

Al respecto, existe una vertiente de desarrollo de modelos de \gls{ia} para la generación de audio, como {Stable Audio} \citep{Audio}, {MusicML} \citep{MusicLM} de Google y {AWS DeepComposer} \citep{AWSDeepComposer} de Amazon. Dichos modelos generan audio a partir de texto natural, mostrando resultados muy prometedores ---particularmente para terrenos relacionados con el marketing y la creación de contenido--- por la potencial reducción de costes de producción. Sin embargo, los archivos o flujos de audio generados son inalterables y escapan al control humano, lo cual limita su aplicación en composición musical donde, entendemos, la intervención humana es esencial\footnote{En nuestro estudio se partirá del postulado de que el factor creador humano es esencial para que un producto sea considerado o no arte.}. En contraste, los \gls{llm} generan texto, incluso texto musical, lo que permite la edición y el control por parte del compositor, abriendo, así, amplias posibilidades de interacción humano-máquina en el proceso creativo.

A pesar de que se han realizado muchos estudios sobre la capacidad de los \gls{llm} para generar algoritmos en lenguajes de programación enfocados en la resolución de problemas matemáticos y el desarrollo de {software}, campo en el que los \gls{llm} han mostrado una elevada efectividad, existe una carencia en lo que respecta al estudio de la generación de código musical con finalidad artística y su alineación con las intenciones del creador sonoro o compositor. No es equiparable, en este sentido, desarrollar, por ejemplo, un código de una función matemática con crear una obra sonora o musical. Al contrario de lo que se podría intuir en un primer momento, el segundo caso requiere, con creces, de un nivel superior de comprensión, abstracción, conocimientos previos y conexión con la realidad que el primero, aunque su código pueda ser eventualmente más simple que el de la función matemática. Por ello, resulta fundamental realizar estudios específicos que exploren la capacidad de los \gls{llm} en la generación de código musical con finalidad artística así como los modos de potenciarla.

Un aspecto muy importante a investigar a la hora de estudiar la utilidad e influencia de sistemas de \gls{ia} en la creación artística es el de su \emph{interfaz} de comunicación con el artista. La interfaz es el medio por el cual artista y sistema interactúan, y puede ser de muy diversa índole. En el caso de los \gls{llm}, la interfaz más común es la {chatbot}, donde la comunicación se establece en forma de diálogo en lenguaje natural. Sin embargo, existen otras interfaces que pueden ser más adecuadas para la creación musical, como las gráficas, interfaces de voz o gestuales, etc. Las modernas herramientas de \gls{ia} ofrecen la posibilidad de crear fácilmente interfaces personalizadas con el uso de \gls{api}, lo que abre un amplio abanico de posibilidades creativas en el ámbito musical y en la investigación.

A pesar de la existencia de numerosos lenguajes de programación dedicados a la creación musical que son relevantes para este estudio, nos enfocaremos específicamente en dos debido a limitaciones de espacio. El primero es \emph{SuperCollider} \citep{SuperCollider2024}, un lenguaje de programación consolidado para la síntesis de sonido y el procesamiento de audio, elegido por su expresividad y su reconocimiento dentro de la comunidad de música electrónica y experimental. El segundo, \emph{Tidal Cycles} \citep{TidalCycles}, destaca por su enfoque en la improvisación sonora y el \emph{live coding}, lo que facilita la exploración de interfaces novedosas. La combinación entre las habilidades de codificación de los \gls{llm} y la flexibilidad de estos lenguajes abre un abanico de posibilidades creativas. Este estudio busca explorarlas, con el objetivo de entender cómo la creatividad generada por los \gls{llm} puede influir en la composición musical algorítmica guiada por un compositor humano. Se espera que los hallazgos puedan aplicarse a otros lenguajes de programación orientados a la creación sonora o musical, como {Overtone} \citep{OvertoneCollaborativeProgrammable}, {Pure Data} \citep{PureDataPd} y {Sonic Pi} \citep{SonicPiLive}, entre otros.

Con este trabajo se busca aportar al mundo de la música y, en concreto, del arte sonoro, una nueva perspectiva sobre la creación musical, apoyada en la investigación en \gls{ia}, para expandir las fronteras de la creatividad y explorar interfaces innovadoras entre el artista y la máquina. Aspiramos a que sea una invitación a la experimentación y la investigación en este campo en rápido desarrollo. 
Asimismo, se espera que la comunidad científica de la \gls{ia} encuentre en este estudio una retroalimentación valiosa desde la investigación musical, contribuyendo a mejorar los modelos de \gls{ia} y sus aplicaciones en el campo artístico. Igualmente, este trabajo quiere incentivar la investigación y desarrollo de \gls{llm} \emph{open source}, lo cual, eventualmente, fomentará la colaboración futura entre las comunidades científica y artística en la creación de tecnologías abiertas para la creación musical y sonora.