\section{Objetivos de la investigación}
\label{sec:objetivos}

% \defaultFontEpigraph{Attention Is All You Need}{\cite{vaswaniAttentionAllYou2017}}

\defaultFontEpigraph{Attention Is All You Need}{Vaswani et al. (2017)}

\textbf{Objetivo general:} Analizar de forma integral la interacción con Modelos de Lenguaje a Gran Escala (LLM) en el ámbito de la creación de música algorítmica en SuperCollider y Tidal Cycles, centrándose en aspectos como la diversidad de interfaces, la adecuación de estos modelos a entradas en lenguaje natural y su influencia en las decisiones creativas durante el proceso compositivo y performativo.

\textbf{Objetivos específicos:}
\begin{enumerate}[label=\alph*)]
    \item Examinar la habilidad de los \gls{llm} para generar código de programación musical basado en texto en lenguaje natural.
    \item Evaluar la eficiencia y usabilidad de distintas interfaces de interacción entre el compositor y los \gls{llm} en la creación sonora.
    \item Explorar el impacto de las diversas técnicas de \emph{prompting} en los resultados producidos por sistemas de \gls{llm}.
    \item Reflexionar\todo{En lugar de "determinar"} sobre el valor artístico de las composiciones y códigos musicales generados mediante la interacción con \gls{llm}.
    \item Considerar el impacto\todo{en lugar de Reflexionar sobre la influencia...} del uso de sistemas de \gls{ia} en el proceso creativo y en la experiencia perceptual del artista.
\end{enumerate}

