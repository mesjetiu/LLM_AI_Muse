\section{Consideraciones iniciales}

El presente trabajo utiliza la norma APA 6ª edición para la citación de referencias bibliográficas. Todas las citaciones a lo largo del texto utilizan hipervínculos hacia sus entradas correspondientes en \emph{Referencias bibliográficas} de la página~\pageref{chap:referencias}. Gran parte de los términos científicos utilizados se encuentran en inglés y no se han traducido al español, ya que son ampliamente conocidos y usados en su forma original inglesa en el ámbito científico hispanohablante. De forma consistente, todos ellos están escritos en cursiva. Los términos utilizados repetidamente a lo largo del texto se encuentran en forma de acrónimos, los cuales son explicitados en su primer uso. El significado de estos acrónimos puede consultarse en el glosario de la página~\pageref{chap:glosario}, al cual apuntan todos ellos por medio de hipervínculos.

Eventualmente, se incluyen códigos QR que enlazan a vídeos, audios u otros recursos de interés (por ej., en la Figura \ref{fig:chatgpt_zero_shot_vs_few_shot}). Estos códigos pueden ser escaneados con cualquier dispositivo móvil, aunque también incluyen enlaces para su acceso desde el propio lector de PDF. Estos recursos son accesorios al texto y pueden cambiar o desaparecer con el tiempo. Por ello, en el Anexo \ref{anexo:repositorio} se incluye un enlace al repositorio de \emph{Github} donde se aloja el código fuente de este trabajo, y la versión actualizada de este PDF.

Por último, el título de este trabajo, hace un guiño a las ya cientos de publicaciones científicas que utilizan la expresión \emph{All You Need} en el título, como se puede observar en la Figura \ref{fig:all_you_need_publicaciones}. Esta expresión se ha convertido en un icono de los avances en inteligencia artificial especialmente desde la conocidísima publicación del artículo \emph{Attention Is All You Need} \citep{vaswaniAttentionAllYou2017} El lector curioso puede encontrar una lista actualizada de estas publicaciones en el repositorio \emph{Awesome "all you need" papers} \citep{nishiKentoNishiAwesomeallyouneedpapers2024}.


\begin{figure}[H]
    \caption[Número de publicaciones científicas del campo de la inteligencia artificial por año que contienen la expresión <<All You Need>> en su título]{Número de publicaciones científicas del campo de la inteligencia artificial por año que contienen la expresión <<All You Need>> en su título.}
    \centering
    \includegraphics[width=0.7\textwidth]{./figuras/all_you_need_publicacionies_anuales.png}
    \source{\propio\ a partir del listado de \cite{nishiKentoNishiAwesomeallyouneedpapers2024}}
    \label{fig:all_you_need_publicaciones}
\end{figure}


% Nos encontramos en una era tecnológicamente acelerada donde la investigación se vuelve esencial, a pesar de que pueda quedar obsoleta en un corto período. En unos pocos años, será impensable un ámbito de la vida humana que quede fuera del alcance de la \gls{ia}. Es esperable que todo profesional, incluso en campos artísticos, incorpore estas herramientas en su trabajo. Por otra parte, cualquier retroalimentación al \gls{dl}, por mínima que sea, en este caso desde el ámbito musical, se presenta como necesaria para comprender mejor su funcionamiento.

% Queda fuera de este estudio cualquier evaluación ética o antropológica de la \gls{ia}, más allá de una perspectiva práctica y descriptiva del uso de los \gls{llm} en la creación musical y sonora, asumiendo que, al momento de esta investigación, los \gls{llm} pueden razonar y crear a niveles que se acercan al de los humanos. Como toda nueva tecnología, ha de madurar en el colectivo humano y encontrar su lugar de integración en lo laboral, lo artístico, lo ético y lo legal, para así tener una fotografía más amplia de lo que la \gls{ia} significa.
