\chapter{Justificación}
\todo{Revisar texto para que quede claro por qué este trabajo es oportuno para la comunidad científica --música, tecnología-- y para mí personalmente.}

% \defaultFontEpigraph{All You Need Is Beyond a Good Init}{\cite{xieAllYouNeed2017}}
\defaultFontEpigraph{All You Need Is Beyond a Good Init}{Xie, Xiong, y Pu (2017)}

La evolución de la \gls{ia}, específicamente en el ámbito del \gls{dl}, ha impulsado el desarrollo de Modelos de Lenguaje a Gran Escala (\glsdisp{llm}{LLM}, por sus siglas en inglés, \textit{Large Language Models}) basados en técnicas avanzadas de aprendizaje profundo. Estos modelos, cuyo primer objetivo fue modelizar el lenguaje humano, han exhibido capacidades emergentes como la creatividad, el razonamiento y habilidades de programación avanzadas. Esto ha impactado rápidamente en campos tan variados como la literatura, la programación, el arte visual, la investigación, el marketing, la docencia y, en definitiva, en la mayor parte de las disciplinas que utilizan el lenguaje y el razonamiento.

Con sorpresa para la comunidad científica, los \gls{llm} han demostrado ser particularmente efectivos en la generación de código de programación, lo que incluye lenguajes enfocados en la generación musical o sonora, siempre y cuando exista suficiente material en la web para entrenar a estos modelos. Sin embargo, la generación de música mediante representaciones simbólicas, como el formato MIDI —el estándar dominante en representación musical digital—, presenta desafíos significativos para los \gls{llm}. Generar música en dicho formato requiere un conocimiento profundo de teoría musical, armonía, contrapunto y la habilidad de razonar sobre estos elementos, lo cual no se puede dar por supuesto en los \gls{llm} de propósito general, a finales de 2023, no especializados en estas áreas.

En este panorama, no se puede ignorar la investigación en el \gls{dl} relativa al desarrollo de modelos de audio, como \textit{Stability Audio}, \textit{MusicML} de Google o \textit{AWS DeepComposer} de Amazon. Estos modelos generan música (audio) a partir de texto natural. Los resultados son prometedores, especialmente en ámbitos como el marketing, por el eventual abaratamiento de costes en la producción. No obstante, estos archivos de audio generados no son editables y escapan al control humano, lo cual limita su uso en la composición musical donde la intervención humana es esencial. Por otro lado, los LLM generan texto a partir de texto. La música representada en código informático es esencialmente texto, a pesar de significar un mundo sonoro. El texto generado por los LLM puede ser editado y controlado por el humano, abriendo así enormes posibilidades de interacción que se exploran en este trabajo.

Existen investigaciones centradas en la generación, por parte de estos \gls{llm}, de algoritmos en lenguajes de programación donde se espera un resultado matemático preciso. Sin embargo, hay un vacío en lo que respecta a la búsqueda de un resultado artístico alineado con las intenciones del creador de sonido o compositor. No es lo mismo crear un código de una función que eleve al cuadrado cualquier entrada numérica, que crear una función que genere <<el sonido de un bosque en otoño>>. El nivel de comprensión, abstracción, conocimientos previos y contacto con la realidad debe ser considerablemente mayor en el segundo caso, incluso cuando su código sea eventualmente más sencillo que el de una función matemática.

Como se verá, existen numerosos lenguajes de programación enfocados en la creación musical que podrían ser relevantes para este trabajo. No obstante, la investigación se limitará a experimentar con unos pocos de ellos por razones de espacio. El más importante será \textit{SuperCollider}, un lenguaje consolidado para síntesis sonora y procesamiento de audio, que ofrece un entorno propicio para esta investigación debido a su riqueza expresiva y su reconocimiento en la comunidad de música electrónica y experimental. La fusión de las capacidades de codificación de los LLM con la versatilidad de \textit{SuperCollider} augura un panorama lleno de posibilidades creativas. Este estudio explora dicho panorama, intentando discernir cómo se manifiesta la creatividad emergente de los LLM en la composición musical algorítmica y cómo esta interacción puede expandir las fronteras de la música digital. Los resultados de este trabajo serán extrapolables a cualquier lenguaje estructurado orientado a la creación sonora o musical, como \textit{Overtone}, \textit{Pure Data} y \textit{Sonic Pi}, entre otros. \todo{Reformar para hablar también de Tidal Cycles}

Nos encontramos en una era tecnológicamente acelerada donde la investigación se vuelve esencial, a pesar de que pueda quedar obsoleta en un corto período. En unos pocos años, será impensable un ámbito de la vida humana que quede fuera del alcance de la \gls{ia}. Es esperable que todo profesional, incluso en campos artísticos, incorpore estas herramientas en su trabajo. Por otra parte, cualquier retroalimentación al \gls{dl}, por mínima que sea, en este caso desde el ámbito musical, se presenta como necesaria para comprender mejor su funcionamiento.

Queda fuera de este estudio cualquier evaluación ética o antropológica de la \gls{ia}, más allá de una perspectiva práctica y descriptiva del uso de los \gls{llm} en la creación musical y sonora, asumiendo que, al momento de esta investigación, los \gls{llm} pueden razonar y crear a niveles que se acercan al de los humanos. Como toda nueva tecnología, ha de madurar en el colectivo humano y encontrar su lugar de integración en lo laboral, lo artístico, lo ético y lo legal, para así tener una fotografía más amplia de lo que la \gls{ia} significa.
