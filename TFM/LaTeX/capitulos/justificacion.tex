\chapter{Justificación}
\todo{Revisar texto para que quede claro por qué este trabajo es oportuno para la comunidad científica --música, tecnología-- y para mí personalmente.}

% \defaultFontEpigraph{All You Need Is Beyond a Good Init}{\cite{xieAllYouNeed2017}}
\defaultFontEpigraph{All You Need Is Beyond a Good Init}{Xie, Xiong, y Pu (2017)}

La evolución de la \gls{ia}, específicamente en el ámbito del \gls{dl} o <<Aprendizaje Profundo>>, ha impulsado el desarrollo de Modelos de Lenguaje a Gran Escala (\glsdisp{llm}{LLM}, por sus siglas en inglés, \textit{Large Language Models}) basados en técnicas avanzadas de aprendizaje profundo. Estos modelos, cuyo primer objetivo fue modelizar el lenguaje humano, han exhibido capacidades emergentes como la creatividad, el razonamiento y habilidades de programación avanzadas. Esto ha impactado rápidamente en campos tan variados como la literatura, la programación, el arte visual, la investigación, el marketing, la docencia y, en definitiva, en la mayor parte de las disciplinas que utilizan el lenguaje y el razonamiento.

Con sorpresa para la comunidad científica, los \gls{llm} han demostrado ser particularmente efectivos en la generación de código de programación, lo que incluye lenguajes enfocados en la generación musical o sonora, siempre y cuando exista suficiente material en la web para entrenar a estos modelos. Sin embargo, la generación de música mediante representaciones simbólicas (aquellas que no son sonido por sí mismas), presenta desafíos significativos para los \gls{llm}. Generar música en dicho formato requiere un conocimiento profundo de teoría musical, armonía, contrapunto y la habilidad de razonar sobre estos elementos, lo cual en este momento no se puede dar por supuesto en los \gls{llm} de propósito general, no especializados en estas áreas.

En este panorama, no se puede ignorar la investigación en el \gls{dl} relativa al desarrollo de modelos de audio, como \textit{Stability Audio}, \textit{MusicML} de Google o \textit{AWS DeepComposer} de Amazon. Estos modelos generan música (audio) a partir de texto natural. Los resultados son prometedores, especialmente en ámbitos como el marketing, por el eventual abaratamiento de costes en la producción. No obstante, estos archivos de audio generados no son editables y escapan al control humano, lo cual limita su uso en la composición musical donde la intervención humana es esencial. Por otro lado, los LLM generan texto a partir de texto, que sí puede ser editado y controlado por el compositor humano, abriendo así enormes posibilidades de interacción que se exploran en este trabajo.

Existen investigaciones centradas en la generación, por parte de estos \gls{llm}, de algoritmos en lenguajes de programación para resolver problemas matemáticos y la creación de \textit{software}, campo donde han mostrado una altísima efectividad. Sin embargo, hay un vacío en lo que respecta a la búsqueda de un resultado artístico alineado con las intenciones del creador de sonido o compositor. No es lo mismo crear un código de una función que eleve al cuadrado cualquier entrada numérica, que crear una función que genere <<el sonido de un bosque en otoño>>. El nivel de comprensión, abstracción, conocimientos previos y contacto con la realidad debe ser considerablemente mayor en el segundo caso, incluso cuando su código sea eventualmente más sencillo que el de una función matemática.

Por otra parte, no solo las potencialidades de los \gls{llm} son objeto de interés y de investigación en la creación artística y musical. También es relevante la interfaz que une al sistema de \gls{ia} con el trabajo del artista humano. En los últimos meses se han popularizado las interfaces tipo \textit{chat}, pero la ductilidad de estos sistemas y la posibilidad de ser usados desde aplicaciones externas abren un gran camino a la investigación. Este trabajo explorará diversas interfaces entre los \gls{llm} y el compositor, creando herramientas ad hoc.

Como se verá, existen numerosos lenguajes de programación enfocados en la creación musical que podrían ser relevantes para este trabajo. No obstante, la investigación se limitará a experimentar con dos de ellos por razones de espacio. El más importante será \textit{SuperCollider}, un lenguaje consolidado para síntesis sonora y procesamiento de audio, que ofrece un entorno propicio para esta investigación debido a su riqueza expresiva y su reconocimiento en la comunidad de música electrónica y experimental. El segundo, \textit{Tidal Cycles}, parte de la filosofía de la improvisación sonora en interpretaciones en vivo o \textit{Live Coding}, lo cual permite la exploración de interfaces novedosas. La fusión de las capacidades de codificación de los LLM con la versatilidad de estos lenguajes augura un panorama lleno de posibilidades creativas. Este estudio explora dicho panorama, intentando discernir cómo se manifiesta la creatividad emergente de los LLM en la composición musical algorítmica y cómo esta interacción puede expandir las fronteras de la música digital. Los resultados de este trabajo serán extrapolables a cualquier lenguaje estructurado orientado a la creación sonora o musical, como \textit{Overtone}, \textit{Pure Data} y \textit{Sonic Pi}, entre otros.

Este trabajo espera aportar al mundo de la música y del arte en general una nueva perspectiva sobre la creación musical y sonora, apoyada en la investigación en \gls{ia}, para expandir las fronteras de la creatividad y explorar interfaces novedosas entre el artista y la máquina.Esperamos que resulte en una invitación a la experimentación y a la investigación en este campo, que se encuentra en un rápido desarrollo.

De igual modo, se espera que la comunidad científica encuentre en este trabajo un \textit{feedback} útil desde el ámbito de la investigación musical, que permita mejorar los modelos de \gls{ia} y sus aplicaciones en el mundo real. Por otra parte, este trabajo puede ser un incentivo más para la investigación y desarrollo de los \gls{llm} \textit{open source}, los cuales permitirán a la comunidad científica y artística aunar esfuerzos en la investigación y desarrollo de tecnologías abiertas para la creación musical y sonora.


