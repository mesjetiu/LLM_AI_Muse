\chapter{Justificación}

% \defaultFontEpigraph{All You Need Is Beyond a Good Init}{\cite{xieAllYouNeed2017}}
\defaultFontEpigraph{All You Need Is Beyond a Good Init}{Xie, Xiong, y Pu (2017)}

La evolución de la (\gls{ia}), específicamente en el dominio del Aprendizaje Profundo (\gls{dl}), ha fomentado el desarrollo de Modelos de Lenguaje a Gran Escala (\glsdisp{llm}{LLM}, por sus siglas en inglés, \textit{Large Language Models}). Estos modelos, originalmente diseñados para emular el lenguaje humano, han demostrado capacidades emergentes en áreas como la creatividad, el razonamiento y habilidades avanzadas de programación. Su influencia se extiende a sectores diversos como la literatura, la programación, el arte visual, la investigación, el marketing y la educación, afectando prácticamente todas las disciplinas que involucran el uso del lenguaje y el razonamiento.

Contrario a las expectativas iniciales, los \gls{llm} han probado ser especialmente eficaces en la generación de código de programación, incluyendo lenguajes relacionados con la creación musical o sonora, siempre que dispongan de material suficiente en la web para su entrenamiento. No obstante, la generación de música mediante representaciones simbólicas (no sonoras en sí mismas) plantea retos significativos para los \gls{llm}. Tal generación musical requiere un conocimiento profundo de teoría musical, armonía y contrapunto, además de la capacidad de razonar sobre estos conceptos, lo cual actualmente no es una premisa en los \gls{llm} de uso general, no especializados en música.

En este contexto, es imposible ignorar la investigación en \gls{dl} relativa al desarrollo de modelos de audio como \textit{Stability Audio}, \textit{MusicML} de Google y \textit{AWS DeepComposer} de Amazon. Dichos modelos generan música (audio) a partir de texto natural, mostrando resultados prometedores, particularmente en marketing por la potencial reducción de costos de producción. Sin embargo, los archivos de audio generados son inalterables y escapan al control humano, lo cual limita su aplicación en composición musical donde la intervención humana es crucial. En contraste, los LLM generan texto a partir de texto, permitiendo edición y control por parte del compositor, abriendo así amplias posibilidades de interacción que este trabajo investiga.

Se han realizado estudios sobre la capacidad de los \gls{llm} para generar algoritmos en lenguajes de programación enfocados en la resolución de problemas matemáticos y el desarrollo de \textit{software}, mostrando una elevada efectividad. No obstante, existe una carencia en lo que respecta a la generación de resultados artísticos alineados con las intenciones del creador sonoro o compositor. No es lo mismo desarrollar un código que eleve al cuadrado una entrada numérica, que crear una función que genere <<el sonido de un bosque en otoño>>. El segundo caso requiere de un nivel superior de comprensión, abstracción, conocimientos previos y conexión con la realidad, incluso si su código puede ser eventualmente más simple que el de una función matemática.

Además, no solo las capacidades de los \gls{llm} son de interés en la creación artística y musical, sino también la interfaz que conecta el sistema de \gls{ia} con el trabajo del artista humano. Recientemente, se han popularizado interfaces tipo \textit{chat}, pero la flexibilidad de estos sistemas y su integración en aplicaciones externas abren un amplio campo de investigación. Este trabajo explora diversas interfaces entre los \gls{llm} y el compositor, desarrollando herramientas específicas.

Aunque existen numerosos lenguajes de programación enfocados en la creación musical relevantes para este estudio, la investigación se centrará en dos de ellos por limitaciones de espacio. El principal será \textit{SuperCollider}, un lenguaje consolidado para síntesis sonora y procesamiento de audio, que ofrece un entorno propicio para esta investigación por su riqueza expresiva y reconocimiento en la comunidad de música electrónica y experimental. El segundo, \textit{Tidal Cycles}, se basa en la filosofía de la improvisación sonora en interpretaciones en vivo o \textit{Live Coding}, lo cual permite la exploración de interfaces innovadoras. La combinación de las capacidades de codificación de los LLM con la versatilidad de estos lenguajes promete un panorama lleno de posibilidades creativas. Este estudio investiga dicho panorama, buscando comprender cómo la creatividad emergente de los LLM se manifiesta en la composición musical algorítmica y cómo esta interacción puede ampliar las fronteras de la música digital. Los resultados serán aplicables a cualquier lenguaje estructurado orientado a la creación sonora o musical, como \textit{Overtone}, \textit{Pure Data} y \textit{Sonic Pi}.

Este trabajo busca aportar al mundo de la música y el arte una nueva perspectiva sobre la creación musical y sonora, apoyada en la investigación en \gls{ia}, para expandir las fronteras de la creatividad y explorar interfaces innovadoras entre el artista y la máquina. Aspiramos a que sea una invitación a la experimentación y la investigación en este campo en rápido desarrollo.

Asimismo, se espera que la comunidad científica encuentre en este estudio retroalimentación valiosa desde la investigación musical, contribuyendo a mejorar los modelos de \gls{ia} y sus aplicaciones prácticas. Además, este trabajo puede incentivar la investigación y desarrollo de \gls{llm} \textit{open source}, facilitando la colaboración entre la comunidad científica y artística en la creación de tecnologías abiertas para la creación musical y sonora.