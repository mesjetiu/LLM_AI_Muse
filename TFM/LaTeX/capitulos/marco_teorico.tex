\chapter{Marco teórico y estado de la cuestión}

% \defaultFontEpigraph{A Broad Dataset Is All You Need\dots}{\cite{michaelisBroadDatasetAll2022}}
\defaultFontEpigraph{A Broad Dataset Is All You Need}{Michaelis, Bethge, y Ecker (2022)}

La estructura de este marco teórico presenta los fundamentos de la \gls{ia}, el \gls{ml} y el \gls{dl}, y avanza hacia los \gls{llm}. Esta progresión, junto a los conceptos que involucra, proporciona el conocimiento necesario para comprender el trascurso de nuestra investigación en los capítulos \ref{chap:lenguajes}, \ref{chap:interfaces_openai}, \ref{chap:ai_muse} y \ref{chap:algorai}. Finalmente, el estado de la cuestión se centrará de forma específica en la generación musical y la generación de código de programación con \gls{ia}, en cuya intersección se sitúa el objeto del presente estudio.


% \section{Inteligencia artificial, aprendizaje automático y aprendizaje profundo}
\import{./capitulos/marco_teorico/}{ia_ml_dl.tex}

% \section{Modelos de lenguaje}
\import{./capitulos/marco_teorico/}{modelos_lenguaje.tex}

% \section{\emph{Prompting engineering}}
\import{./capitulos/marco_teorico/}{prompting_engineering.tex}

% \section{Limitaciones conocidas de los LLM}
\import{./capitulos/marco_teorico/}{limitaciones_llm.tex}

% Estado de la cuestión
\import{./capitulos/marco_teorico/}{estado_cuestion.tex} 







