\chapter{Metodología}

\begin{itemize}
    \item Revisión bibliográfica:
    Revisar los artículos científicos más recientes y significativos relacionados con modelos de lenguaje a gran escala, basados en la arquitectura Transformer, enfocándose en su aplicación en campos no lingüísticos, como es la composición musical.
    \item Configuración y preparación de herramientas:
    Selección inicial del modelo: Inicio de la experimentación con GPT-4 de OpenAI dadas sus capacidades avanzadas para programación, dejando abierta la posibilidad de incorporar otros modelos si surgen durante el proceso.
    \item Experimentación:
    Proyectos guiados: Desarrollar ideas y proyectos musicales con la ayuda del modelo seleccionado.
    \item Experimentación abierta:
    Modificar y adaptar códigos preexistentes, como los de SuperCollider, para explorar propuestas y resultados del modelo.
    \item Utilización de la interfaz de ChatGPT (OpenAI):
    Uso complementario para obtener resultados y comparar con la interfaz personalizada.
    \item Registro detallado:
    Cada paso, interacción y resultado del proceso quedará documentado para garantizar un seguimiento completo de la experiencia.
    \item Evaluación de los resultados:
    Analizar la calidad, originalidad y aplicabilidad de las propuestas y resultados generados.
    \item Reflexión sobre la interacción: Considerar la eficiencia y productividad de la colaboración con el modelo en comparación con procesos creativos sin su intervención.
\end{itemize}



\section{Enfoque, alcance y diseño}

\section{Variables}

\section{Desarrollo y aplicación}
