\chapter{Metodología}

\section{Revisión Bibliográfica o Estado del Arte}

\subsection{Objetivos de la Revisión}
Esta revisión tiene como propósito comprender el estado actual de los Modelos de Lenguaje a Gran Escala (LLM) y identificar las mejores prácticas en cuanto a la interacción mediante "prompting" en programación informática y composición musical.

\subsection{Fuentes y Búsqueda}
Para llevar a cabo esta revisión, se consultaron diversas bases de datos, revistas especializadas y conferencias de relevancia en el ámbito de la inteligencia artificial, programación y música. Las palabras clave utilizadas para la búsqueda incluyeron términos como "LLM", "prompting", "programación informática", "composición musical", entre otros.

\subsection{Selección y Evaluación}
Los criterios utilizados para seleccionar los trabajos incluidos en esta revisión se centraron en la relevancia y actualidad. Se dieron prioridad a trabajos publicados en los últimos cinco años y aquellos que se centran específicamente en aspectos cruciales de los LLM y su interacción con el proceso compositivo.

\subsection{Resultados de la Revisión}
Los hallazgos más relevantes de esta revisión serán discutidos en detalle en el capítulo correspondiente al Estado del Arte. No obstante, es importante destacar que existen múltiples enfoques y técnicas para interactuar con los LLM, y que su aplicación en la composición musical aún se encuentra en una fase exploratoria.

\section{Enfoque, alcance y diseño}
El enfoque de esta investigación es principalmente cualitativo, aunque se incorporarán elementos cuantitativos para evaluar ciertas métricas específicas. El alcance se centra en la interacción con los LLM en el contexto de la composición musical algorítmica, utilizando SuperCollider como principal herramienta. El diseño de la investigación implica un proceso iterativo de experimentación con diferentes prompts y análisis de los resultados obtenidos.

\section{Variables}
Aunque el enfoque es primordialmente cualitativo, se considerarán variables cuantitativas como:
\begin{itemize}
    \item Número de prompts utilizados.
    \item Porcentaje de respuestas correctas según los parámetros definidos.
    \item Calidad artística del output (evaluación subjetiva).
\end{itemize}

\section{Desarrollo y Aplicación}

\subsection{Herramientas y Recursos Utilizados}
Para llevar a cabo la experimentación en esta investigación, se utilizará el modelo GPT-4 proporcionado por OpenAI a través de su API. GPT-4 es, a día de hoy, el modelo de lenguaje a gran escala más avanzado y potente disponible públicamente. Esta elección garantiza la interacción con una tecnología de vanguardia en el campo del procesamiento del lenguaje natural y el deep learning. La API de OpenAI permite una interacción flexible y eficiente con el modelo, facilitando la introducción de prompts y la recopilación de respuestas en tiempo real.

\subsection{Experimentación con Prompts}
Se realizarán múltiples sesiones de experimentación, en las cuales se introducirán diversos prompts tanto de sistema como de usuario para interactuar con el LLM.

\subsection{Análisis Cuantitativo}
Una vez recopilados los resultados, se llevará a cabo un análisis cuantitativo para determinar métricas como el porcentaje de respuestas correctas y la calidad artística del output.

\subsection{Análisis Cualitativo y Reflexión}
Además del análisis cuantitativo, se reflexionará sobre las interacciones, las respuestas obtenidas y la experiencia durante el proceso, proporcionando hallazgos cualitativos sobre la interacción con los LLM.

\subsection{Comparación y Reflexión Final}
Se realizará una reflexión comparativa sobre la influencia de los LLM en el proceso creativo en contraste con creaciones realizadas sin su intervención. También se discutirán las implicaciones, limitaciones y posibles aplicaciones futuras de los descubrimientos de la investigación.