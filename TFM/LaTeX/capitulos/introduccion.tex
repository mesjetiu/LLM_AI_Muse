\chapter{Introducción}

\section{Justificación}

La evolución de la inteligencia artificial, específicamente dentro del ámbito del \textit{Deep Learning}, ha impulsado el desarrollo de Modelos de Lenguaje a Gran Escala (LLM, por sus siglas en inglés) basados en técnicas avanzadas de aprendizaje profundo. Estos modelos han exhibido capacidades emergentes, como manifestar creatividad y habilidades de programación avanzadas, y han impactado campos tan variados como la literatura, la programación y el arte visual.

Los LLM han demostrado ser particularmente efectivos en la generación de código de programación, incluyendo lenguajes enfocados en la generación musical, siempre que exista suficiente material de entrenamiento. Sin embargo, la generación de música mediante representaciones simbólicas, como el formato MIDI —el estándar dominante en representación musical digital—, presenta desafíos para los LLM. Generar música en dicho formato demanda un conocimiento profundo de teoría musical, armonía, contrapunto y la habilidad de razonar sobre estos elementos, lo cual es un desafío para los LLM de propósito general no especializados en estas áreas.

Existen investigaciones centradas en la generación de algoritmos en lenguajes de programación donde se espera un resultado matemático preciso. Sin embargo, hay un vacío cuando lo que se busca, más allá de la corrección sintáctica, es un resultado artístico alineado con las intenciones del creador de sonido o compositor.

\textit{SuperCollider}, un lenguaje consolidado para síntesis sonora y procesamiento de audio, ofrece un entorno propicio para esta investigación debido a su riqueza expresiva y su reconocimiento en la comunidad de música electrónica y experimental. La fusión de las capacidades de codificación de los LLM con la versatilidad de \textit{SuperCollider} augura un panorama lleno de posibilidades creativas. Este estudio explora dicho panorama, intentando discernir cómo se manifiesta la creatividad emergente de los LLM en la composición musical algorítmica y cómo esta interacción puede expandir las fronteras de la música digital.

Nos encontramos en una era de aceleración en la que la investigación se vuelve esencial, aunque también pueda quedar obsoleta en un corto período. En unos pocos años, será impensable un ámbito intelectual no influenciado por la IA. Se espera que todo profesional, incluso en campos artísticos, incorpore estas herramientas en su trabajo. Además, cualquier retroalimentación al \textit{Deep Learning}, por mínima que sea, es de un gran valor dada su capacidad de crecimiento exponencial.

Este estudio no proporciona una evaluación ética o antropológica de la IA, sino una perspectiva práctica y descriptiva del uso de los LLM en la creación musical y sonora, asumiendo que, al momento de esta investigación, los LLM pueden razonar y crear a niveles comparables a los humanos.

Aunque \textit{SuperCollider} ha sido el lenguaje elegido para este estudio, los resultados son extrapolables a cualquier lenguaje estructurado orientado a la creación sonora o musical, como \textit{Overtone}, \textit{Pure Data} o \textit{Sonic Pi}. La elección de \textit{SuperCollider} se debe a su madurez, aceptación en la comunidad y familiaridad para los LLM actuales.


\section{Objetivos de la investigación}

\textbf{Objetivo general:}
\begin{itemize}
    \item Describir la interacción con modelos de lenguaje a gran escala en la composición musical algorítmica, considerando la corrección de sus resultados, la capacidad de ajustar su salida a inputs en lenguaje natural y su potencialidad para influir en decisiones creativas.
\end{itemize}

\textbf{Objetivos específicos:}
\begin{itemize}
    \item Analizar el proceso y resultado de la interacción con los LLM en la creación musical algorítmica en términos de eficiencia, corrección formal y capacidad de comprensión del lenguaje natural en la generación de elementos musicales.
    \item Evaluar la contribución creativa de los LLM al proceso compositivo y en qué medida estos modelos puede potenciar y expandir la creatividad y la obra final.
    \item Investigar la capacidad de razonamiento de los LLM al incorporar conocimientos, teorías o técnicas de otras áreas en el ámbito de la música experimental.
    \item Capturar y reflexionar sobre la experiencia emocional y perceptual durante la interacción con los LLM: Entender cómo el uso de estos modelos afecta el sentimiento, la intuición y la percepción del proceso creativo y del resultado final.
    \item Reflexionar de manera comparativa sobre la influencia de los LLM en el proceso creativo en contraste con creaciones realizadas sin su intervención.
\end{itemize}
