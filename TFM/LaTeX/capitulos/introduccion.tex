\chapter{Introducción}

Sin duda alguna, se trata de un tema candente que impregna todo, incluso la música.

Modelos de Ia generativa música...

LLM, los más sorprendentes, porque razonan, tal como dice \cite{chenTeachingLargeLanguage2023}.

Tienen habilidades emergentes, incluyendo la creatividad as artística.

Los LLM se han mostrado muy efectivos en la generación de código de programación, sin excepción de lenguajes orientados a la generación musical. 

Existen trabajos enfocados en la generación de algoritmos en lenguajes de programación, en el que lo que se espera de una función es un resultado matemático, pero aún poco o nada si lo que se espera más allá de la corrección sintáctica es un resultado artístico que apoye las intenciones del creador de sonido o compositor.

Este trabajo se enmarca en un momento vertiginoso, en el que toda investigación es necesaria, al tiempo que está condenada a la obsolescencia en pocos meses. Pero es claro que en un plazo de pocos años no habrá campo intelectual en el que la IA no haya entrado. No se entenderá un profesional de cualquier campo, incluido el artístico, que trabaje al margen de estas herramientas.

No se encontrará aquí ninguna valoración de índole ética o antropológica de la IA, sino más bien una valoración práctica y descriptiva del uso de los LLM en la creación musical en general, y sonora en particular, dando como un hecho que los LLM son capaces de razonar y crear a niveles comparables al humano en el momento de este estudio.

Aunque se ha elegido SuperCollider como el lenguaje objeto del estudio, la investigación es extrapolable a cualquier lenguaje estructurado cuyo destino sea la creación sonora o musical, como Overtone, Pure Data, Sonic Pi... El hecho de que Supercollider es un lenguaje maduro y muy extendido entre la comunidad, al tiempo que razonablemente bien conocido por los LLM actualmente, ha sido uno de los motivos de su elección.

Podria haber sido más atractivo un estudio sobre como los LLM construyen composiciones en un lenguaje más clásico como puede ser el MIDI, si bien, esto claramente dista bastante de lo que estos sistemas son capaces de hacer a día de hoy sin un entrenamiento ad hoc, ya que la complejidad de una composición con notas musicales recae en disciplinas como la armonía, el contrapunto, etc, donde los LLM de propósito general no han sido entrenados lo suficiente a día de hoy.

\section{Objetivos de la investigación}

\textbf{Objetivo general:}
\begin{itemize}
    \item Describir y reflexionar sobre la interacción personal con modelos de lenguaje a gran escala (LLM) en el proceso de composición musical, considerando la autonomía del LLM y su capacidad para influir en decisiones creativas.
\end{itemize}

\textbf{Objetivos específicos:}
\begin{itemize}
    \item Documentar el proceso y resultado de la interacción con los LLM en la creación musical algorítmica: Analizar la eficiencia y valor de las iteraciones al generar elementos musicales y reflexionar de manera comparativa sobre la influencia de los LLM en el proceso creativo en contraste con creaciones realizadas sin su intervención.
    \item Evaluar la contribución creativa de los LLM al proceso compositivo: Reflexionar sobre cómo la interacción con estos modelos puede potenciar y expandir la creatividad y la obra final.
    \item Explorar las capacidades emergentes y creativas de los LLM: Profundizar en cómo los LLM manifiestan habilidades emergentes durante el proceso de composición y cómo estos modelos pueden trascender sus patrones de entrenamiento para ofrecer propuestas innovadoras.
    \item Investigar la capacidad de razonamiento de los LLM al incorporar conocimientos externos: Analizar cómo estos modelos pueden entender, procesar e integrar teorías o técnicas de otras áreas en el ámbito de la música experimental, y reflexionar sobre las implicaciones creativas de esta incorporación en la composición.
    \item Capturar y reflexionar sobre la experiencia emocional y perceptual durante la interacción con los LLM: Entender cómo el uso de estos modelos afecta el sentimiento, la intuición y la percepción del proceso creativo y del resultado final.
\end{itemize}
