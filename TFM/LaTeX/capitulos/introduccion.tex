\chapter{Introducción}

% \defaultFontEpigraph{All You Need Is a Good Init}{\cite{mishkinAllYouNeed2016}}
\defaultFontEpigraph{All You Need Is a Good Init}{Mishkin y Matas (2016)}


Este Trabajo de fin de máster se centra en analizar de forma integral la interacción con \gls{llm} en el ámbito de la creación de música algorítmica mediante los lenguajes de programación musical de SuperCollider y Tidal Cycles. Se abordará, en concreto, cómo estos modelos, al ser capaces de generar código de programación a partir de texto en lenguaje natural, influyen en diversos aspectos del proceso creativo. Así mismo, se evaluará la eficiencia y usabilidad de distintas interfaces de interacción entre el compositor y los \gls{llm}, así como el impacto de las técnicas de \emph{prompting} en los resultados de composición. Además, se reflexionará sobre el valor artístico de las obras generadas y el efecto de estos sistemas de \gls{ia} en el proceso creativo y la percepción subjetiva del artista humano creador, proporcionando así un entendimiento profundo de su papel en la innovación y evolución de la música algorítmica.

La aproximación metodológica será de carácter cualitativo, en consonancia con el enfoque exploratorio y descriptivo del estudio. De una forma práctica y como parte y producto de este estudio, se ha creado un software para la generación de código musical en vivo, \emph{AI Muse}, que se presenta en el capítulo \ref{chap:ai_muse}, y que queda a disposición de la comunidad científica como software libre, y una obra de arte sonoro, \emph{AlgorAI}, sobre la cual tratamos en el capítulo \ref{chap:algorai}.

La intención final de este trabajo es aportar conocimiento sobre la interacción entre el ser humano y las herramientas modernas de \gls{iag}, desde un punto de vista artístico y creativo. Al mismo tiempo, se pretende ofrecer una visión crítica y reflexiva sobre el papel real que en este momento pueden desempeñar sistemas construidos sobre tecnologías de \gls{llm}, y apuntar hacia futuras líneas de investigación musical con \gls{ia}.



\import{./capitulos/introduccion/}{consideraciones.tex}
\import{./capitulos/introduccion/}{justificacion.tex}
\import{./capitulos/introduccion/}{objetivos.tex}