\chapter{Introducción}

% \defaultFontEpigraph{All You Need Is a Good Init}{\cite{mishkinAllYouNeed2016}}
\defaultFontEpigraph{All You Need Is a Good Init}{Mishkin y Matas (2016)}


Este trabajo de fin de máster se centra en analizar de forma integral la interacción humano-máquina, con grandes modelos de lenguaje, o \glsdisp{llm}{\emph{large language models} (LLM)}, en el ámbito de la creación de música algorítmica mediante los lenguajes de programación musical de SuperCollider y Tidal Cycles. Se abordará, en concreto, cómo estos modelos de lenguaje, al ser capaces de generar código de programación a partir de texto en lenguaje natural, influyen en diversos aspectos del proceso creativo. Asimismo, se explorará la eficiencia y usabilidad de distintas interfaces de interacción entre el compositor y los \gls{llm}, así como el impacto de las técnicas de \emph{prompting}\footnote{Entendemos por \emph{prompt} el mensaje o conjunto de palabras que se envían como input a un sistema de inteligencia artificial para que genere una respuesta. \emph{Prompting}, en sentido amplio, podemos concebirlo como el arte o la ciencia de crear estos \emph{prompts}. Se estudiará con más detalle en la sección \ref{sec:llm_tecnicas_prompting}.} en los resultados de composición. Además, se reflexionará sobre el valor artístico de las obras generadas, el efecto de estos sistemas de \gls{ia} en el proceso creativo y la percepción del artista humano, proporcionando así claves para un entendimiento profundo el papel de los modernos \gls{llm} en la innovación y evolución de la música algorítmica.

La metodológica será cualitativa, en consonancia con el enfoque exploratorio y descriptivo del estudio, lo cual tiene la virtud de ofrecer un panorama global del objeto de estudio, aunque a cambio de una menor precisión en los resultados. De una forma práctica y como parte y producto de este estudio, se ha creado un software para la generación de código musical en vivo, \emph{AI Muse}, que se presenta en el capítulo \ref{chap:ai_muse}, y que queda a disposición de la comunidad científica como software libre (véase anexo \ref{anexo:repositorio}); y una obra de arte sonoro, \emph{AlgorAI}, compuesta a partir de los resultados del estudio, tratada en el capítulo \ref{chap:algorai}.

La intención final de este trabajo es aportar conocimiento sobre la interacción entre el ser humano y las herramientas modernas de \gls{iag}, desde un punto de vista creativo. Al mismo tiempo, se pretende ofrecer una visión crítica y reflexiva sobre el papel real que en este momento pueden desempeñar sistemas construidos sobre tecnologías de \gls{llm}, y apuntar hacia futuras líneas de investigación musical con \gls{ia}.



\import{./capitulos/introduccion/}{consideraciones.tex}
\import{./capitulos/introduccion/}{justificacion.tex}
\import{./capitulos/introduccion/}{objetivos.tex}