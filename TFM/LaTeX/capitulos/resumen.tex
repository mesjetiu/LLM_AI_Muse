  % Página "Resumen"
  \clearpage
  \chapter*{Resumen}

  Este estudio se enfoca en la interacción entre Modelos de Lenguaje a Gran Escala (LLM) y la creación de música algorítmica, analizando cómo estos modelos pueden influir en el proceso creativo a través de diversas interfaces humano-máquina. Empleando una metodología exploratoria, se evalúa la capacidad de los LLM para generar código de programación musical en distintos contextos, especialmente en entornos de \emph{live coding} con herramientas como \emph{SuperCollider} y \emph{Tidal Cycles}. Como parte de la investigación, se ha desarrollado \emph{AI Muse}, un software específicamente diseñado para optimizar la ejecución de código en actuaciones en vivo, promoviendo una interacción eficaz y directa con los LLM. La composición de \emph{AlgorAI}, una obra de arte sonoro realizada durante el estudio, ejemplifica tanto el potencial creativo como las limitaciones de los LLM, destacando la imprescindible colaboración humano-máquina en la formulación de estructuras complejas. Los resultados del estudio resaltan las limitaciones actuales de los LLM en la generación de código sonoro complejo, mientras ilustran su importante papel como asistentes creativos, herramientas de exploración y fuentes de inspiración.
  


  \vspace{1cm}
  \textbf{Palabras clave:} % palabras clave en español. Máximo 5 palabras

  Modelos de lenguaje a gran escala, aprendizaje profundo, creación musical, interacción humano-máquina, \emph{live coding}.


    % Página "Abstract"
    \clearpage
    \chapter*{Abstract}
  
    This study focuses on the interaction between Large Language Models (LLMs) and the creation of algorithmic music, examining how these models can influence the creative process through various human-machine interfaces. Employing an exploratory methodology, it evaluates the ability of LLMs to generate musical programming code in different contexts, particularly within live coding environments using tools such as SuperCollider and Tidal Cycles. As part of the research, \emph{AI Muse} was developed, software specifically designed to optimize code execution in live performances, facilitating effective and direct interaction with LLMs. The composition of \emph{AlgorAI}, a sound art piece created during the study, exemplifies both the creative potential and the limitations of LLMs, highlighting the essential collaboration between humans and machines in developing complex structures. The findings of the study underscore the current limitations of LLMs in generating complex sound code, while illustrating their significant role as creative assistants, exploration tools, and sources of inspiration.

    
    \vspace{1cm}
    \textbf{Keywords:} % Keywords in English. Maximum 5 words
    
    Large Language Models, Deep Learning, Music Creation, Human-Machine Interaction, Live Coding.
  