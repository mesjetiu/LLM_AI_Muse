  % Página "Resumen"
  \clearpage
  \chapter*{Resumen}

  Este trabajo se centra en la interacción integral de Modelos de Lenguaje a Gran Escala (LLM) con la creación de música algorítmica, examinando cómo estos modelos pueden influir en el proceso creativo. La investigación profundiza en la capacidad de los LLM para generar código de programación musical en diversas interfaces humano-máquina. \emph{AI Muse} es una herramienta específicamente diseñada para este estudio, que optimiza la ejecución de código en entornos de \emph{live coding} con SuperCollider y Tidal Cycles, facilitando una comunicación fluida y directa. El estudio explora la generación de estructuras sonoras y patrones rítmicos a través de un proceso iterativo que busca maximizar la precisión y la creatividad de las respuestas de los LLM. \emph{AlgorAI} es una obra de arte sonoro compuesta en la investigación, e ilustra tanto el potencial creativo como las limitaciones de los LLM, resaltando la importancia de la colaboración humano-máquina para superar desafíos en la generación de estructuras complejas. Los resultados indican que los LLM pueden desempeñar un rol significativo como asistentes creativos, herramientas de exploración y fuentes de inspiración.


  \vspace{1cm}
  \textbf{Palabras clave:} % palabras clave en español. Máximo 5 palabras

  Modelos de lenguaje a gran escala, aprendizaje profundo, creación musical, interacción humano-máquina, \emph{live coding}.


    % Página "Abstract"
    \clearpage
    \chapter*{Abstract}
  
    This study delves into the comprehensive interaction between Large Language Models (LLMs) and the creation of algorithmic music, investigating the impact these models can have on the creative process. It examines the capability of LLMs to produce musical programming code across various human-machine interfaces. \emph{AI Muse}, a tool specifically developed for this research, enhances code execution in live coding environments utilizing SuperCollider and Tidal Cycles, thereby ensuring smooth and direct communication. The research focuses on the iterative generation of sound structures and rhythmic patterns, aiming to optimize both the accuracy and creativity of LLM responses. \emph{AlgorAI}, a piece of sound art created during the study, showcases the creative potential and limitations of LLMs, emphasizing the crucial role of human-machine collaboration in overcoming challenges associated with generating complex structures. The findings indicate that LLMs can serve as creative assistants, exploration tools, and sources of inspiration.
  
    \vspace{1cm}
    \textbf{Keywords:} % Keywords in English. Maximum 5 words
    
    Large Language Models, Deep Learning, Music Creation, Human-Machine Interaction, Live Coding.
  