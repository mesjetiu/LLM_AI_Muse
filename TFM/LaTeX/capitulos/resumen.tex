  % Página "Resumen"
  \clearpage
  \chapter*{Resumen}

  Este trabajo se centra en la interacción integral de Modelos de Lenguaje a Gran Escala (LLM) con la creación de música algorítmica, examinando cómo estos modelos pueden influir en el proceso creativo. La investigación profundiza en la capacidad de los LLM para generar código de programación musical en diversas interfaces humano-máquina. \emph{AI Muse} es una herramienta específicamente diseñada para este estudio, que optimiza la ejecución de código en entornos de \emph{live coding} con SuperCollider y Tidal Cycles, facilitando una comunicación fluida y directa. El estudio explora la generación de estructuras sonoras y patrones rítmicos a través de un proceso iterativo que busca maximizar la precisión y la creatividad de las respuestas de los LLM. \emph{AlgorAI} es una obra de arte sonoro compuesta durante la investigación, e ilustra tanto el potencial creativo como las limitaciones de los LLM, resaltando la importancia de la colaboración humano-máquina para superar desafíos en la generación de estructuras complejas. Este trabajo ofrece una evaluación crítica del uso de LLM en la composición musical algorítmica y abre nuevas perspectivas hacia ulteriores investigaciones fomentando la colaboración entre comunidades científicas y artísticas y avanzando en el uso de tecnologías abiertas para la creación musical.

  \vspace{1cm}
  \textbf{Palabras clave:} % palabras clave en español. Máximo 5 palabras

  Modelos de lenguaje a gran escala, aprendizaje profundo, creación musical, interacción humano-máquina, \emph{live coding}.


    % Página "Abstract"
    \clearpage
    \chapter*{Abstract}
  
    This work focuses on the comprehensive interaction of Large Language Models (LLMs) with algorithmic music creation, examining how these models can influence the creative process. The research delves into the ability of LLMs to generate musical programming code across various human-machine interfaces. \emph{AI Muse} is a tool specifically designed for this study, optimizing code execution in live coding environments with SuperCollider and Tidal Cycles, facilitating fluid and direct communication. The study explores the generation of sound structures and rhythmic patterns through an iterative process that seeks to maximize the precision and creativity of LLM responses. \emph{AlgorAI} is a sound art piece composed during the research, illustrating both the creative potential and limitations of LLMs, highlighting the importance of human-machine collaboration to overcome challenges in generating complex structures. This work offers a critical assessment of the use of LLMs in algorithmic musical composition and opens new perspectives towards further research by fostering collaboration between scientific and artistic communities and advancing the use of open technologies for musical creation.
  
    \vspace{1cm}
    \textbf{Keywords:} % Keywords in English. Maximum 5 words
    
    Large Language Models, Deep Learning, Music Creation, Human-Machine Interaction, Live Coding.
  