\chapter{Objetivos de la investigación}

\textbf{Objetivo general:} Examinar la interacción de modelos de lenguaje a gran escala en la creación sonora a través de lenguajes de programación musicales, con el foco en variedad de interfaces y aplicaciones, su adaptabilidad a entradas en lenguaje natural y la capacidad de los modelos para influir en decisiones creativas.

\textbf{Objetivos específicos:}
\begin{enumerate}[label=\alph*)]
\item Evaluar la eficiencia, corrección formal y comprensión del lenguaje natural de los LLM en la generación de elementos musicales algorítmicos.
\item Investigar el impacto creativo de los LLM en el proceso compositivo, tanto de piezas como de interpretaciones en vivo, determinando cómo la interacción con estos modelos de lenguaje puede potenciar y expandir tanto la creatividad como el producto musical final.
\item Analizar el razonamiento de los LLM al integrar conceptos, teorías o técnicas de otras disciplinas en el contexto de la música experimental.
\item Crear un conjunto de scripts que permitan la interacción de los LLM con lenguajes de programación musicales, con el fin de explorar las capacidades de los LLM en la creación musical y sonora.
\item Contraponer y reflexionar sobre la influencia de los LLM en el proceso creativo, en comparación con composiciones realizadas sin su intervención, así como la experiencia emocional y perceptual del artista durante el proceso.
\end{enumerate}
