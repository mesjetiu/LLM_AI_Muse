\chapter{Conclusiones}

% \defaultFontEpigraph{All You Need Is Boundary}{\cite{wangAllYouNeed2019}}
\defaultFontEpigraph{All You Need Is Boundary}{H. Wang et al. (2019)}


Este trabajo ha explorado la interacción entre la \gls{ia} y la generación algorítmica de música, investigando tanto las posibilidades como las limitaciones de los \gls{llm} en este ámbito creativo. Mediante un enfoque cualitativo que combina experimentación práctica con análisis teórico, se ha buscado evaluar el impacto de los \gls{llm} en la composición musical.

A continuación, se detallan las principales conclusiones alcanzadas en este trabajo, organizadas en función de los objetivos específicos establecidos (véase la sección \ref{sec:objetivos}). Para cada uno, se proporciona una evaluación del grado de cumplimiento y las implicaciones de los hallazgos:

\begin{enumerate}[label=\alph*)]
    \item \textit{Examinar la habilidad de los \gls{llm} para generar código de programación musical basado en texto en lenguaje natural:} Este objetivo ha sido el motor transversal de toda la investigación, ya que toda interacción entre humano y \gls{llm} se ha examinado a través del lenguaje natural, propio de los \gls{ml}. Los resultados (cfr. capítulo \ref{chap:resultados}) dan amplia cuenta tanto de las capacidades como de las limitaciones halladas en la generación de código musical.
    % hecho
    

    \item \textit{Evaluar la eficiencia y usabilidad de interfaces de interacción entre el compositor y los \gls{llm} en la creación sonora:} A lo largo del trabajo se han evaluado diversas interfaces de comunicación e interacción compositor-\gls{llm}, desde el diálogo con un chatbot hasta la utilización creativa de una \gls{api} para administrar programáticamente las peticiones a un \gls{llm} y procesar sus respuestas en un entorno de codificación musical en vivo. El estudio de esta última interfaz, además, ha dado como fruto la creación de un programa de \emph{live coding} que permite la interacción en tiempo real con un \gls{llm} a través de la \gls{api} de OpenAI (véase sección \ref{sec:generacion_automatica_codigo_tidal_cycles} y capítulo \ref{chap:ai_muse}).
    % hecho
    

    \item \textit{Explorar el impacto de las diversas técnicas de \emph{prompting} en los resultados producidos por sistemas de \gls{llm}:} En cada fase de la investigación, se experimentó con diferentes formas de prompting. Especial atención merecen los prompts de sistema utilizados en los programas de \emph{live coding}, que constituyen el conjunto de instrucciones fundamentales sobre el que los \gls{llm} actúan en la creación musical.
    % hecho
    
    
    \item \textit{Reflexionar sobre el valor artístico de las composiciones y códigos musicales generados mediante la interacción con \gls{llm}:}  El trabajo destaca la capacidad de estos sistemas para producir texturas sonoras y composiciones que, aunque no siempre estructuradas como piezas completas, ofrecen una base rica para la creación musical. Se han estudiado las condiciones en las que la interacción con sistemas de \gls{llm} puede conducir a la creación de música con calidad artística. La composición realizada en el trabajo, \emph{AlgorAI}, muestra un posible camino de integración de la generación de código musical por \gls{llm} en una obra de arte sonoro completa.
    % hecho
    
    \item \textit{Considerar el impacto del uso de sistemas de \gls{ia} en el proceso creativo y en la experiencia perceptual del artista:} La búsqueda de interfaces y modos de comunicación con sistemas de \gls{ia} en este trabajo ha sido un proceso de reflexión constante sobre el papel de los \gls{llm} en el proceso creativo, no solo desde un punto de vista del objeto final generado, sino también desde la perspectiva de la experiencia del artista en el proceso de creación. Así, por una parte, la creación de \emph{AlgorAI}, como obra musical, y por otra, la escritura de \emph{AI Muse}, han explorado la percepción del propio artista-investigador en todos los procesos implicados.
\end{enumerate}
