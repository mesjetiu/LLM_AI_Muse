\chapter{Conclusiones}
\todo{conclusiones que permite
conocer si se han conseguido los objetivos (según la rúbrica)}

% \defaultFontEpigraph{All You Need Is Boundary}{\cite{wangAllYouNeed2019}}
\defaultFontEpigraph{All You Need Is Boundary}{H. Wang y cols. (2019)}


Este estudio ha explorado la intersección entre la \gls{ia} y la generación algorítmica de música, destacando tanto las capacidades como las limitaciones de los \gls{llm} en la interacción con el proceso creativo musical humano, explorando desde la generación de código de programación musical hasta el impacto en la percepción en el propio proceso creativo del artista. A través de un enfoque metodológico cualitativo que combinó la experimentación práctica con el análisis teórico, se buscó comprender y evaluar la capacidad de los \gls{llm} en el ámbito de la composición musical. 


A continuación, se presentan las conclusiones principales derivadas del análisis integral de la interacción entre el compositor y los LLM, planteado en el objetivo general, reflejando cómo cada objetivo específico ha sido abordado y en qué medida se han cumplido. 

\begin{enumerate}[label=\alph*)]
    \item \textbf{Examinar la habilidad de los \gls{llm} para generar código de programación musical basado en texto en lenguaje natural:} Este objetivo ha sido el motor transversal de toda la investigación, ya que toda interacción entre humano y \gls{llm} se ha examinado a través del lenguaje natural, propio de los \gls{ml}. Los resultados (cfr. capítulo \ref{chap:resultados}) dan amplia cuenta tanto de las capacidades como de las limitaciones halladas en la generación de código musical.
    % hecho
    

    \item \textbf{Evaluar la eficiencia y usabilidad de interfaces de interacción entre el compositor y los \gls{llm} en la creación sonora:} A lo largo del trabajo se han evaluado diversas interfaces de comunicación e interacción compositor-\gls{llm}, desde el diálogo con un chatbot hasta la utilización creativa de una API para administrar programáticamente las peticiones a un \gls{llm} y procesar sus respuestas en un entorno de codificación musical en vivo. El estudio de esta última interfaz ha dado como fruto la creación de un programa de live coding que permite la interacción en tiempo real con un \gls{llm} a través de la API de OpenAI (véase sección \ref{sec:generacion_automatica_codigo_tidal_cycles} y capítulo \ref{chap:ai_muse}).
    % hecho
    

    \item \textbf{Explorar el impacto de las diversas técnicas de \emph{prompting} en los resultados producidos por sistemas de \gls{llm}:} La investigación evidenció cómo el diseño adecuado de técnicas de prompting puede dirigir y mejorar la originalidad y calidad de la música generada, cumpliendo así con el tercer objetivo específico. Se observó que los prompts pueden influir significativamente en la improvisación colaborativa y en los resultados musicales obtenidos.
    \item \textbf{Determinar el valor artístico de las composiciones y códigos musicales generados mediante la interacción con \gls{llm}:} A través del proceso de selección y composición, se demostró que los materiales generados por LLM poseen un valor artístico considerable, alineándose con el cuarto objetivo específico. El trabajo destacó la capacidad de estos sistemas para producir texturas sonoras y composiciones que, aunque no siempre estructuradas como piezas completas, ofrecen una base rica para la creación musical.
    \item \textbf{Reflexionar sobre la influencia del uso de sistemas de \gls{ia} en el proceso creativo y en la experiencia emocional y perceptual del artista:} La interacción con LLM permitió al compositor explorar nuevas texturas y timbres, y facilitó un proceso compositivo en el que la selección y combinación de materiales sonoros quedaron a cargo de decisiones creativas humanas, en consonancia con el quinto objetivo específico. Esta experiencia subraya el potencial transformador de la IA en la creación musical, permitiendo al artista expandir sus horizontes creativos más allá de lo que sería posible sin esta tecnología.
\end{enumerate}
