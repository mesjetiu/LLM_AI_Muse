\chapter{Conclusiones}
\todo{conclusiones que permite
conocer si se han conseguido los objetivos (según la rúbrica)}

% \defaultFontEpigraph{All You Need Is Boundary}{\cite{wangAllYouNeed2019}}
\defaultFontEpigraph{All You Need Is Boundary}{H. Wang y cols. (2019)}


Este estudio ha explorado la intersección entre la \gls{ia} y la generación algorítmica de música, destacando tanto las capacidades como las limitaciones de los \gls{llm} en la interacción con el proceso creativo musical humano, explorando desde la generación de código de programación musical hasta el impacto en la percepción en el propio proceso creativo del artista. A través de un enfoque metodológico cualitativo que combinó la experimentación práctica con el análisis teórico, se buscó comprender y evaluar la capacidad de los \gls{llm} en el ámbito de la composición musical. 


A continuación, se presentan las conclusiones principales derivadas del análisis integral de la interacción entre el compositor y los LLM, planteado en el objetivo general, reflejando cómo cada objetivo específico ha sido abordado y en qué medida se han cumplido. 

\begin{enumerate}
    \item \textbf{Capacidad de los \gls{llm} para generar código de programación musical a partir de texto en lenguaje natural:} La automatización y la fluidez en los diálogos con el sistema LLM demostraron una notable capacidad de estos modelos para interpretar instrucciones en lenguaje natural y traducirlas en código de programación musical funcional. Este logro responde directamente al primer objetivo específico, facilitando la creación musical algorítmica y permitiendo una interacción que puede ser percibida como una colaboración humano-máquina.
    
    \item \textbf{Eficiencia y usabilidad de interfaces de interacción:} La experiencia de uso mejorada, centrada únicamente en cuestiones musicales sin distracciones, refleja el cumplimiento del segundo objetivo específico. Se destacó la importancia de los prompts de sistema bien diseñados para una comunicación efectiva con el LLM, subrayando la relevancia de interfaces intuitivas en el proceso creativo.
    
    \item \textbf{Impacto de las técnicas de prompting:} La investigación evidenció cómo el diseño adecuado de técnicas de prompting puede dirigir y mejorar la originalidad y calidad de la música generada, cumpliendo así con el tercer objetivo específico. Se observó que los prompts pueden influir significativamente en la improvisación colaborativa y en los resultados musicales obtenidos.
    
    \item \textbf{Valor artístico de las composiciones generadas:} A través del proceso de selección y composición, se demostró que los materiales generados por LLM poseen un valor artístico considerable, alineándose con el cuarto objetivo específico. El trabajo destacó la capacidad de estos sistemas para producir texturas sonoras y composiciones que, aunque no siempre estructuradas como piezas completas, ofrecen una base rica para la creación musical.
    
    \item \textbf{Influencia en el proceso creativo y experiencia del artista:} La interacción con LLM permitió al compositor explorar nuevas texturas y timbres, y facilitó un proceso compositivo en el que la selección y combinación de materiales sonoros quedaron a cargo de decisiones creativas humanas, en consonancia con el quinto objetivo específico. Esta experiencia subraya el potencial transformador de la IA en la creación musical, permitiendo al artista expandir sus horizontes creativos más allá de lo que sería posible sin esta tecnología.
\end{enumerate}