\chapter{Entorno de trabajo y experimentación}
    \section{Elección del modelo de LLM e hiperparámetros óptimos}
        \subsection{Modelo de LLM}
            Elección de GPT-4.
        \subsection{Hiperparámetros}
            Temperatura, top\_p, ventana de contexto, número de tokens, etc.
    \section{Entorno de trabajo}
        Distinguir entre el entorno de pruebas en chatGPT y el entorno delimitado de la API de OpenAI.
        \subsection{Entorno de pruebas con chatbot}
            Útil para experimentación libre. Pero no hay control de los hiperparámetros, las especificaciones pueden cambiar, lo que compromete la repibilidad de los resultados. No permite prompts en 2-steps.
        \subsection{Entorno de trabajo con la API en Playground y en Python}
            Hablar de LangChain y de la API de OpenAI.

\chapter{Estrategias de \textit{prompting} para el diseño sonoro}
    \section{\textit{Chain of Thoughts} y \textit{Structured Chain of Thoughts}}
    \section{\textit{1-step} o \textit{2-step}}
    \section{Uso de \textit{Self-debugging} y \textit{Chain of Verification}}

\chapter{\textit{System Prompts} y \textit{User Prompts} diseñados para esta investigación}
    \section{Diseños de \textit{System Prompts}}
        Han de ser Pocos. Quizás entre 3 y 5. Describir los que se han elegido y lo que se espera de ellos.
    \section{Diseños de \textit{User Prompts}}
        Los prompts de usuario escogidos para solicitar al modelo la generación de sonidos y texturas sonoras. Los prompts no deben ser muchos, para poder llevar a cabo la investigación en un tiempo razonable. Deben abarcar diveras técnicas de síntesis, la creación de diversas texturas descritas en lenguaje natural, y la aplicación de teoremas matemáticos, principios físicos, o inspiracion en otras artes, como la pintura o la literatura para la generación sonora.



