\chapter{\emph{AI Muse}, herramienta en \emph{Python} para la interacción en vivo LLM en entornos de \emph{live coding}}
\label{chap:ai_muse}

% \defaultFontEpigraph{One Model Is All You Need}{\cite{grahamOneModelAll2022}}
\defaultFontEpigraph{One Model Is All You Need}{Graham y cols. (2022)}

En este capítulo pasamos a describir la herramienta \emph{AI Muse}, a la cual nos referimos en el capítulo anterior (véase \ref{sec:generacion_automatica_codigo_tidal_cycles}), detallando su funcionamiento, diversas cuestiones de su implementación y los resultados de su utilización.

\section{Descripción general de la herramienta}

\emph{AI Muse} es una herramienta de interacción con modelos de lenguaje en entornos de live coding. Está escrita en Python y se comunica con la API de OpenAI. Permite ejecutar código de SuperCollider y de Tidal Cycles. Además, tiene un archivo de configuración en formato JSON, así como un conjunto de comandos para modificar en vivo los parámetros de la configuración. Aunque está pensada para la utilización de modelos de OpenAI, esta librería permite eventualmente comunicarse con otros modelos de lenguaje incluso en local, detalle que no ha sido explorado en este trabajo.

% Escrito en Python
% Se comunica con la API de OpenAI
% Es un entorno de Liveconding
% Al guardar el archivo se realiza un intercambio con la API
% Permite ejecutar código de SuperCollider y código de Tidal Cycles
% Tiene un archivo de configuración en json, además de un conjunto de comandos para cambiar los parámetros en vivo

\section{¿Por qué en entorno de Livecoding?}

El live coding es el arte de programar música en tiempo real. Es un entorno de experimentación y de creación artística que une el campo de la programación con el de la interpretación musical y la improvisación. Por su naturaleza, esta práctica ofrece un entorno ideal para la exploración de las posibilidades que puede ofrecer la interacción con modelos de lenguaje de forma automatizada. 

El live coding, por su naturaleza improvisatoria, no está ceñido a una forma musical predeterminada. Esta indeterminación formal tiene menos exigencias que la composición a nivel temporal, algo que se adecua muy bien a los resultados que se obtuvieron en etapas anteriores de nuestro trabajo, donde se observó que los modelos de lenguaje actuales producen mejores resultados con bloques de código relativamente pequeños. Este formato creativo, unido a las bondades ya señaladas de Tidal Cycles en cuanto a seguridad del código y resistencia a errores, llevó a la escritura \emph{AI Muse} como una herramienta de experimentación.

% Es más dúctil e interactivo que el de la composición.
% Explora la posibilidad de respuesta inmediata por API.
% Explota el hecho de que los LLM actualmente trabajan mejor con bloques relativamente pequeños.
% Los lenguajes de livecoding tienen mecanismos de seguridad que abstraen al usuario de largas comprobaciones de errores: o funciona o no funciona. Si no funciona, nada se cae. Este principio se aplica especialmente a Tidal Cycles, aunque SuperCollider también tiene estructuras adecuadas para livecoding (placeholders)
% Inmediatez de la respuesta del modelo. Se iteran las respuestas buenas, las malas se pueden desechar sin más.

\section{Integración de la API de OpenAI}

\emph{AI Muse} integra dos posibles interacciones que OpenAI permite con su API: utilización modelos de chatbot y de sistemas de \gls{rag}\footnote{Los llamados \emph{Assistants} por OpenAI.}. La primera modalidad es la más sencilla, ya que se trata de realizar conversaciones con el modelo. En este caso, cada petición se limita a un prompt general de sistema unido al archivo actual con todo el código generado, de forma que el modelo pueda genera el siguiente bloque de código.

La diferencia crucial con scripts previos en este trabajo es que es el propio usuario quien decide cuándo hacer una petición a la API, en lugar de estar automatizado. La acción elegida para hacer esta petición es la de guardar el archivo. De esta forma, el programa puede ser utilizado en cualquier editor de textos, y no está limitado a un entorno de desarrollo concreto ni al uso de \emph{plugins} o extensiones. En este caso, todo el programa ha sido escrito y testado en Visual Studio Code, pero eventualmente podría ser utilizado en cualquier otro editor de textos.

% Usar la API es conveniente para controlar todos los parámetros del modelo.
% Actualmente tiene dos modalidades: Chat y Assistant.
% Es lo suficientemente rápido como para interaccionar y escribir mucho más rápido que el usuario, por lo que se puede poner un tiempo mínimo entre respuestas.

\section{Lenguajes de programación musical disponibles}

En un primer momento se implementó únicamente la posibilidad de ejecutar código de Tidal Cycles. Posteriormente, se añadió el soporte para SuperCollider, aprovechando las clases que este lenguaje ofrece para el live coding. En ambos casos se ofrece la posibilidad de que sea el propio programa el que se ocupe de llamar a los binarios de los propios lenguajes, o que este código sea ejecutado por el usuario por otros medios (uso de plugins o de entornos de desarrollo específicos).


\section{Flujo de trabajo}

El flujo de trabajo típico de \emph{AI Muse} es el siguiente:

\begin{enumerate}
    \item El usuario ejecuta el programa. Automáticamente se creará un archivo en texto plano con la extensión adecuada (actualmente, \texttt{.tidal} o \texttt{.sc}, para Tidal Cycles y SuperCollider, respectivamente) para interactuar.
    \item El usuario abre un archivo de código en un editor de texto.
    \item \label{item:paso_write_code} El usuario puede escribir código musical, un comando de sistema o un simple comentario-mensaje para el modelo.
    \item Al guardar el archivo, \emph{AI Muse} lee el archivo y lo envía a la API de OpenAI.
    \item La API de OpenAI devuelve una respuesta con el siguiente bloque de código que es añadido automáticamente al archivo actual.
    \item El usuario ejecuta el nuevo bloque de código a voluntad, pudiendo modificarlo si lo cree oportuno.
    \item El usuario vuelve al paso \ref{item:paso_write_code}.
\end{enumerate}

La Figura xxx ilustra este flujo de trabajo.

\missingfigure{Flujo de trabajo de AI Muse}

\section{Archivo de configuración y comandos}

La Figura \ref{fig:json} muestra el archivo de configuración de \emph{AI Muse}. Este archivo será cargado al ejecutar el programa.

\begin{figure}[H]
    \caption[Archivo de configuración \texttt{config.json} de AI Muse]{Archivo de configuración \texttt{\emph{config.json}} de AI Muse. Los valores de cada clave pueden ser modificados directamente en el archivo o por medio de comandos.}
    \centering
% \begin{mdframed}
\setstretch{1} % Esto establece el interlineado a sencillo
\begin{lstlisting}[language=json, numbers=none]
{
    "mode_tidal_supercollider": "tidal",
    "create_log_file": false,
    "ghci_path": "ghci",
    "only_system_commands": false,
    "supercollider_on": false,
    "sclang_path": "sclang",
    "boot_tidal_path": "/usr/share/haskell-tidal/BootTidal.hs",
    "api_enabled": false,
    "bot_mode": "assistant",
    "assistant_id": "asst_9Km6Fd3xywD4tGGY4yZzSEiw",
    "assistant_retrieval_folder": "./assistants/tidal_livecoding/retrieval_files/",
    "api_key_file": "apikey.txt",
    "system_prompt_file": "./assistants/tidal_livecoding/system_prompts/system_prompt_01",
    "model": "gpt-4-1106-preview",
    "temperature": 1,
    "max_tokens": 256,
    "top_p": 1,
    "frequency_penalty": 0,
    "presence_penalty": 0,
    "wait_time_before_api": 20,
    "wait_time_after_api": 20
}
\end{lstlisting}
% \end{mdframed}
    \source{\propio}
    \label{fig:json}
\end{figure}



\begin{landscape}
    \begin{table}[htbp]
        \centering
        \caption{Descripción de la configuración del archivo \texttt{config.json} y de sus correspondientes comandos.}
        \label{tab:config-description}
        % \setstretch{1}
        \fontsize{9.5pt}{11pt}\selectfont
        \begingroup
        \rowcolors{2}{azul_unir_soft}{white}
        \begin{tabularx}{\linewidth}{lXlcl}
            \toprule
            \rowcolor{azul_unir} % Aplica el color azul_unir al encabezado de la tabla
            \textbf{Clave} & \textbf{Valores posibles} & \textbf{Efecto} & \textbf{Modificable en vivo} & \textbf{Ejemplo de uso en comando} \\
            \midrule
        % Tus filas aquí
        mode\_tidal\_supercollider & tidal/sclang & Modo de operación para TidalCycles o SuperCollider & sí & set mode\_tidal\_supercollider tidal \\
        create\_log\_file & \textcolor{truecolor}{true}/\textcolor{falsecolor}{false} & Activa o desactiva la creación de archivos de registro & sí & set create\_log\_file \textcolor{truecolor}{true} \\
        ghci\_path & Ruta & Ruta al ejecutable ghci & no & - \\
        only\_system\_commands & \textcolor{truecolor}{true}/\textcolor{falsecolor}{false} & Restringe a comandos del sistema & sí & set only\_system\_commands \textcolor{falsecolor}{false} \\
        supercollider\_on & \textcolor{truecolor}{true}/\textcolor{falsecolor}{false} & Enciende o apaga SuperCollider & sí & set supercollider\_on \textcolor{truecolor}{true} \\
        sclang\_path & Ruta & Ruta al ejecutable sclang & no & - \\
        boot\_tidal\_path & \textcolor{pathcolor}{/ruta/a/BootTidal.hs} & Ruta al archivo de arranque de Tidal & no & - \\
        api\_enabled & \textcolor{truecolor}{true}/\textcolor{falsecolor}{false} & Activa o desactiva la API & sí & set api\_enabled \textcolor{truecolor}{true} \\
        bot\_mode & assistant/otro & Establece el modo del bot & sí & set bot\_mode assistant \\
        assistant\_id & ID & ID del asistente & no & - \\
        assistant\_retrieval\_folder & \textcolor{pathcolor}{./ruta/a/carpetas} & Ruta a la carpeta de recuperación & no & - \\
        api\_key\_file & \textcolor{pathcolor}{apikey.txt} & Archivo con la clave de la API & no & - \\
        system\_prompt\_file & \textcolor{pathcolor}{./ruta/a/system\_prompts} & Archivo de prompt del sistema & no & - \\
        model & gpt-4-1106-preview/otro & Modelo de lenguaje utilizado & no & - \\
        temperature & \textcolor{numbercolor}{0.0-1.0} & Temperatura para la generación de texto & sí & set temperature \textcolor{numbercolor}{0.7} \\
        max\_tokens & \textcolor{numbercolor}{Número} & Número máximo de tokens por generación & sí & set max\_tokens \textcolor{numbercolor}{256} \\
        top\_p & \textcolor{numbercolor}{0.0-1.0} & Ajusta la probabilidad de tomar caminos menos probables & sí & set top\_p \textcolor{numbercolor}{0.9} \\
        frequency\_penalty & \textcolor{numbercolor}{Número} & Penalización por frecuencia de tokens & sí & set frequency\_penalty \textcolor{numbercolor}{0.5} \\
        presence\_penalty & \textcolor{numbercolor}{Número} & Penalización por presencia de tokens & sí & set presence\_penalty \textcolor{numbercolor}{0.5} \\
        wait\_time\_before\_api & \textcolor{numbercolor}{Segundos} & Tiempo de espera antes de llamar a la API & sí & set wait\_time\_before\_api \textcolor{numbercolor}{10} \\
        wait\_time\_after\_api & \textcolor{numbercolor}{Segundos} & Tiempo de espera después de llamar a la API & sí & set wait\_time\_after\_api \textcolor{numbercolor}{10} \\
        \bottomrule
    \end{tabularx}
    \endgroup
    \source{\propio}
    \end{table}
    \end{landscape}
    


% Explicar el archivo de configuración y los comandos de configuración en vivo. Hacer una tabla



\section{Hallazgos obtenidos con \emph{AI Muse}}

Se ha conseguido un alto nivel de "colaboración" e interacción con un modelo de lenguaje. La dimensión temporal no se decide de antemano, como en una composición, sino que se desarrolla a partir de modificaciones y aportaciones atómicas. En esto es fuerte GPT-4.

Es un buen banco de pruebas para la experimentación con modelos. 

\section{Limitaciones de la herramienta}

\section{Ejemplos de uso}