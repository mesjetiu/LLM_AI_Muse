\chapter{Limitaciones y prospectiva}

% Ir apuntando aquí las limitaciones y prospectivas.

Trabajos futuros con más de dos lenguajes de programación.

Los LLM tienden a repetir ciertos patrones, por lo que podemos hablar de una personalidad creativa del modelo.

Los LLM, a día de hoy, son útiles para la generación de código de timbres y texturas, pero no para la generación de estructuras temporales compositivas.

Investigar sistemas de agentes autónomos que puedan manejar los errores de código iterativamente

Investigar la posibilidad de entrenamiento con fine tuning de los modelos de lenguaje como GPT-3 y GPT-4 en los lenguajes utilizados.

Con el crecimiento de los LLM, investigar su capacidad de manejo de piezas extensas.

En el futuro, la posibilidad e utilizar agentes es muy sugerente y prometedora, ya que auna la capacidad de manejo de códigos pequeños con el de su unidad y planificación general.

Entre otras cosas, comentar la posibilidad de sesgo de todo el trabajo, en cuanto que el investigador es conocedor del lenguaje de programación y de la música, y por tanto, puede que haya sesgos en la elección de los prompts, en la interpretación de los resultados, etc.

Interesante herramienta para la creación "aleatoria" de sonidos y música, en busca de la serendipia creadora.

Gran herramienta usada para fines educativos y de autodidáctica.

Potencia el trabajo del especialista (como a hombros de gigantes)

El entorno de livecoding, y de interacción en tiempo real con APIs es muy prometedor. La dimensión temporal no se decide de antemano, como en una composición, sino que se desarrolla a partir de modificaciones y aportaciones atómicas. En esto es fuerte GPT-4.

En el futuro, investigar la posibilidad de tener LLM en local y open source, así como la posibilidad de entrenarlos con fine tuning.
