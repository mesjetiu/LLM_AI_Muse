\chapter{Limitaciones y prospectiva}

% \defaultFontEpigraph{Is Bang-Bang Control All You Need?}{\cite{seydeBangBangControlAll2021}}

\defaultFontEpigraph{Is Bang-Bang Control All You Need?}{Seyde y cols. (2021)}


\section{Limitaciones}

% Limitación inherente a la naturaleza cualitativa y exploratoria del propio trabajo. No se pueden esperar resultados experimentales.

% Limitación por la existencia de un único investigador, que puede llevar a sesgos en la interpretación de los resultados.

% Limitación por obsolescencia de las herramientas utilizadas. Los modelos de lenguaje utilizados son los más actuales a día de hoy, pero solo tienen que pasar unos meses para que queden obsoletos.

% La utilización de modelos de código cerrados, en lugar de open source, limita la comprensión de los procesos internos de los modelos de lenguaje.

% La imposibilidad de realizar fine tuning de los modelos de lenguaje utilizados por no tener los recursos computacionales necesarios.

% Entre otras cosas, comentar la posibilidad de sesgo de todo el trabajo, en cuanto que el investigador es conocedor del lenguaje de programación y de la música, y por tanto, puede que haya sesgos en la elección de los prompts, en la interpretación de los resultados, etc.


\section{Prospectiva}


% \item \textbf{Aplicaciones educativas y de asistente:} Se reconoce el valor de estas herramientas para la educación en los lenguajes musicales, ofreciendo una asistencia teórica y práctica de gran utilidad.

% Futuros trabajos con metodología cuantitativa, con métricas de calidad de código, de calidad de música, de originalidad, etc.

% Creación de datasets específicos para entrenar modelos de lenguaje en música y sonido.

% Creación de un buen sistema \gls{rag}, con documentos de código comentados y categorizados, para que el modelo pueda aprender de ellos.

% Trabajos futuros con más de dos lenguajes de programación.

% Investigar sistemas de agentes autónomos que puedan manejar los errores de código iterativamente

% Investigar la posibilidad de entrenamiento con fine tuning de los modelos de lenguaje como GPT-3 y GPT-4 en los lenguajes utilizados.

% Con el crecimiento de los LLM, investigar su capacidad de manejo de piezas extensas.

% En el futuro, la posibilidad e utilizar agentes es muy sugerente y prometedora, ya que auna la capacidad de manejo de códigos pequeños con el de su unidad y planificación general.


% En el futuro, investigar la posibilidad de tener LLM en local y open source, así como la posibilidad de entrenarlos con fine tuning.


% Prospectiva: explorar la multimodalidad en el campo audiovisual.

% Explorar más afondo la posibilidad de utilizar LLM en tiempo real, en el campo del live coding.



