\chapter{Resultados y discusión}
\todo{lo que hay en Resultados y en Limitaciones y prospectivas es aún un conjunto de apuntes. Hay que organizarlos.}

% \defaultFontEpigraph{All You Need Is a Second Look\dots}{\cite{caoAllYouNeed2020}}
\defaultFontEpigraph{All You Need Is a Second Look}{Cao y Zou (2020)}

\section{Resultados}

\section{Discusión crítica de los resultados}
% Estamos lejos de una IA que pueda generar música de forma autónoma con fines musicales y absoluta comprensión del proceso compositivo. Sin embargo, los resultados muestran que es una herramienta muy útil para el experto musical.

% Las interacciones que posibilitan las API son un buen punto de partida para la generación de sonido y música en tiempo real, como en el live coding, siempre que existan buenos mecanismos de abstracción de los errores de código.

% A nivel compositivo y de búsqueda de ideas y nuevos sonidos, se muestra como una herramienta prometedora, siempre más provechosa para el experto en el lenguaje utilizado que para quien lo desconoce.


