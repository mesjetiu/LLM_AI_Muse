\section*{Códigos de materiales sonoros escritos en Tidal Cycles}



\begin{minipage}[t]{.48\textwidth}
    \begin{lstlisting}[style=SuperCollider-IDE, basicstyle=\footnotesize\ttfamily, numbers=none]
(
    // Definir los componentes del paisaje sonoro
    var arroyo, pajaros, viento, hojas;
)          
    \end{lstlisting}
    \captionqranexo{Zero-shot en una petición de creación sonora en SuperCollider.}{https://drive.google.com/file/d/13AMMKehJHzlI8CQUqCqoWMHZ9AI2YgcS/view?usp=drive_link}{audio}
\end{minipage}\hfill
\begin{minipage}[t]{.48\textwidth}
    \begin{lstlisting}[style=SuperCollider-IDE, basicstyle=\footnotesize\ttfamily, numbers=none]
% Código de programación      
    \end{lstlisting}
    \captionqranexo{Few-shot en una petición de creación sonora en SuperCollider. El resultado sonoro es más interesante desde el punto de vista artístico.}{https://drive.google.com/file/d/1C7FDkH706e1OMlcOWrG4qsCoC7Au8GaU/view?usp=drive_link}{audio}
\end{minipage}

\vspace{20pt} % Espacio entre filas

% Segunda fila
\begin{minipage}[t]{.48\textwidth}
    \begin{lstlisting}[style=SuperCollider-IDE, basicstyle=\footnotesize\ttfamily, numbers=none]
(
    // Definir los componentes del paisaje sonoro
    var arroyo, pajaros, viento, hojas;
    // Definir los componentes del paisaje sonoro
    var arroyo, pajaros, viento, hojas;
    // Definir los componentes del paisaje sonoro
    var arroyo, pajaros, viento, hojas;
)          
    \end{lstlisting}
    \captionqranexo{Zero-shot en una petición de creación sonora en SuperCollider.}{https://drive.google.com/file/d/13AMMKehJHzlI8CQUqCqoWMHZ9AI2YgcS/view?usp=drive_link}{audio}
\end{minipage}\hfill
\begin{minipage}[t]{.48\textwidth}
    \begin{lstlisting}[style=SuperCollider-IDE, basicstyle=\footnotesize\ttfamily, numbers=none]
(
    // Definir los componentes del paisaje sonoro
    var arroyo, pajaros, viento, hojas;
    // Definir los componentes del paisaje sonoro
    var arroyo, pajaros, viento, hojas;
)          
    \end{lstlisting}
    \captionqranexo{Few-shot en una petición de creación sonora en SuperCollider. El resultado sonoro es más interesante desde el punto de vista artístico.}{https://drive.google.com/file/d/1C7FDkH706e1OMlcOWrG4qsCoC7Au8GaU/view?usp=drive_link}{audio}
\end{minipage}

\vspace{20pt} % Espacio entre filas

% Tercera fila
\begin{minipage}[t]{.48\textwidth}
    \begin{lstlisting}[style=SuperCollider-IDE, basicstyle=\footnotesize\ttfamily, numbers=none]
    (
        // Definir los componentes del paisaje sonoro
        var arroyo, pajaros, viento, hojas;
    )          
    \end{lstlisting}
    \captionqranexo{Zero-shot en una petición de creación sonora en SuperCollider.}{https://drive.google.com/file/d/13AMMKehJHzlI8CQUqCqoWMHZ9AI2YgcS/view?usp=drive_link}{audio}
\end{minipage}\hfill
\begin{minipage}[t]{.48\textwidth}
    \begin{lstlisting}[style=SuperCollider-IDE, basicstyle=\footnotesize\ttfamily, numbers=none]
    (
        // Definir los componentes del paisaje sonoro
        var arroyo, pajaros, viento, hojas;
    )          
    \end{lstlisting}
    \captionqranexo{Few-shot en una petición de creación sonora en SuperCollider. El resultado sonoro es más interesante desde el punto de vista artístico.}{https://drive.google.com/file/d/1C7FDkH706e1OMlcOWrG4qsCoC7Au8GaU/view?usp=drive_link}{audio}
\end{minipage}
  