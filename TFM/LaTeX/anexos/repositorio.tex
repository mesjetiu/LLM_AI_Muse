\chapter{Repositorio con los materiales de la investigación}
% \addcontentsline{toc}{chapter}{Repositorio del presente trabajo}
\label{anexo:repositorio}



\begin{description}
    \item[URL del Repositorio:] \hfill \\
    \url{https://github.com/mesjetiu/LLM_AI_Muse} \qrcode[height=1.4cm]{https://github.com/mesjetiu/LLM_AI_Muse}
    
    \item[Contenido del Repositorio:] \hfill \\
    Este repositorio contiene todos los materiales utilizados y generados a lo largo de esta investigación. Esto incluye:
    \begin{itemize}
        \item Archivos de \defaultLaTeX{} correspondientes a este PDF.
        \item Scripts de Python utilizados para la interacción con la \gls{api} de OpenAI.
        \item Archivos de SuperCollider y Tidal Cycles con los {prompts} obtenidos en el trabajo.
        \item Otros materiales y recursos utilizados.
    \end{itemize}
    
    \item[Versión de depósito final y actualizaciones:] \hfill \\
    La versión final de este trabajo, la cual es la depositada en la universidad, se identifica claramente en el directorio raíz y se nombra como \textbf{\texttt{TFM\_AI\_Muse\_UNIR\_Deposito\_final.pdf}}. Esta versión representa el estado del trabajo en el momento de su depósito oficial y permanecerá inalterada para conservar el registro de la presentación original. El \emph{commmit} correspondiente tiene, además, la etiqueta \textbf{\texttt{Deposito\_final}}:

    % \url{https://github.com/mesjetiu/LLM_AI_Muse/blob/main/TFM_AI_Muse_Deposito_final.pdf} 
    \qrcode[height=1.4cm]{https://github.com/mesjetiu/LLM_AI_Muse/blob/main/TFM_AI_Muse_UNIR_Deposito_final.pdf}

    Además de la versión de depósito, este repositorio alberga una versión eventualmente actualizada del documento, la cual puede recibir actualizaciones posteriores al depósito oficial. Esta versión se denomina \textbf{\texttt{TFM\_AI\_Muse\_UNIR\_Updated.pdf}} y se encuentra disponible mediante el siguiente enlace permanente:

    % \url{https://github.com/mesjetiu/LLM_AI_Muse/blob/main/TFM_AI_Muse_Updated.pdf} 
    \qrcode[height=1.4cm]{https://github.com/mesjetiu/LLM_AI_Muse/blob/main/TFM_AI_Muse_UNIR_Updated.pdf}

    Las actualizaciones a la versión \textbf{\texttt{TFM\_AI\_Muse\_UNIR\_Updated.pdf}} se documentan detalladamente en los \emph{commits} del repositorio. Esto facilita la trazabilidad de los cambios y asegura que los interesados puedan revisar el historial de modificaciones efectuadas al documento tras su depósito.



    



    % TFM_AI_Muse_UNIR_v1.0.0


\end{description}



% \begin{figure}[H]
%     \caption[Estructura de directorios del repositorio del trabajo]{Estructura de directorios del repositorio del trabajo.}
%     % \centering

%     \singlespace	

% 	\dirtree{%
% 		.1 classes\DTcomment{ 
% 			\begin{minipage}[t]{10cm}
% 				Todas las clases de SuperCollider de .
% 		\end{minipage}}.
% 		.2 GUI\DTcomment{ 
% 			\begin{minipage}[t]{8cm}
% 				Todas las clases, gráficos e imágenes de la interfaz gráfica de usuario.
% 		\end{minipage}}.
% 		.3 images.
% 		.4 panels.
% 		.4 widgets.
% 		.2 modules\DTcomment{ 
% 			\begin{minipage}[t]{8cm}
% 				Las clases que representan a los módulos del Synthi 100.
% 		\end{minipage}}.
% 		.2 SGME\_Settings\DTcomment{ 
% 			\begin{minipage}[t]{8cm}
% 				Archivos de configuración (límites, rangos y otros niveles modificables).
% 		\end{minipage}}.
% 		.2 SynthiGME.sc\DTcomment{ 
% 			\begin{minipage}[t]{8cm}
% 				Clase principal.
% 		\end{minipage}}.
% 		.1 HelpSource\DTcomment{ 
% 			\begin{minipage}[t]{8cm}
% 				Archivos de ayuda y documentación.
% 		\end{minipage}}.
% 		.1 Installation\DTcomment{ 
% 			\begin{minipage}[t]{8cm}
% 				Scripts de ayuda y ejemplos.
% 		\end{minipage}}.
% 		.1 Ladish studio\DTcomment{ 
% 			\begin{minipage}[t]{8cm}
% 				Archivos auxiliares para \emph{Jack} en Linux (prescindibles)
% 		\end{minipage}}.
% 		.1 TouchOSC\DTcomment{ 
% 			\begin{minipage}[t]{8cm}
% 				Layout de ejemplo creado con \emph{TouchOSC}.
% 		\end{minipage}}.
% 		.1 COPYING\DTcomment{ 
% 			\begin{minipage}[t]{8cm}
% 				Archivo con la licencia GNU GPLv3.
% 		\end{minipage}}.
% 		.1 README.md\DTcomment{ 
% 			\begin{minipage}[t]{8cm}
% 				Archivo con la descripción básica de la aplicación.
% 		\end{minipage}}.
% 		.1 SynthiGME.quark\DTcomment{ 
% 			\begin{minipage}[t]{8cm}
% 				Archivo de información del \emph{quark} (para SuperCollider).
% 		\end{minipage}}.
% 	}
%     \source{\propio}
%     \label{fig:arbol_directorios}
% \end{figure}

