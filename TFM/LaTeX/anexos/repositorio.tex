\chapter{Repositorio con los materiales de la investigación}
% \addcontentsline{toc}{chapter}{Repositorio del presente trabajo}
\label{anexo:repositorio}



\begin{description}
    \item[URL del Repositorio:] \hfill \\
    \url{https://github.com/mesjetiu/LLM_SuperCollider} \qrcode[height=1.4cm]{https://github.com/mesjetiu/LLM_SuperCollider}
    
    \item[Contenido del Repositorio:] \hfill \\
    Este repositorio contiene todos los materiales utilizados y generados a lo largo de esta investigación. Esto incluye:
    \begin{itemize}
        \item Archivos de \defaultLaTeX{} correspondientes a este PDF.
        \item Scripts de Python utilizados para la interacción con la \gls{api} de OpenAI.
        \item Archivos de SuperCollider y Tidal Cycles con los {prompts} obtenidos en el trabajo.
        \item Otros materiales y recursos utilizados.
    \end{itemize}
    
    \item[Versión de Entrega Final:] \hfill \\
    La versión correspondiente a la entrega final de este trabajo a UNIR está identificada con la etiqueta \textbf{\texttt{Entrega\_final}} en los commits del repositorio. Y el PDF correspondiente se encuentra en el directorio raíz bajo el nombre \textbf{\texttt{TFM\_AI\_Muse\_UNIR\_v1.0.0.pdf}}

	\item[Versión actualizada de este PDF:] \hfill \\
    Eventuales actualizaciones de este documento, especialmente debidas a modificaciones en los enlaces externos, seguirán el esquema de versionado semántico para reflejar la naturaleza y el alcance de los cambios. Las versiones actualizadas se nombrarán siguiendo el patrón \textbf{\texttt{TFM\_AI\_Muse\_UNIR\_vX.Y.Z.pdf}}, donde \textbf{\texttt{X}}, \textbf{\texttt{Y}}, y \textbf{\texttt{Z}} representan el número de versión mayor, menor y de corrección, respectivamente. Por ejemplo, una actualización menor que corrige enlaces rotos podría nombrarse \textbf{\texttt{TFM\_AI\_Muse\_UNIR\_v1.0.1.pdf}}. Encuentre estas versiones en el directorio raíz del repositorio.


\end{description}



% \begin{figure}[H]
%     \caption[Estructura de directorios del repositorio del trabajo]{Estructura de directorios del repositorio del trabajo.}
%     % \centering

%     \singlespace	

% 	\dirtree{%
% 		.1 classes\DTcomment{ 
% 			\begin{minipage}[t]{10cm}
% 				Todas las clases de SuperCollider de .
% 		\end{minipage}}.
% 		.2 GUI\DTcomment{ 
% 			\begin{minipage}[t]{8cm}
% 				Todas las clases, gráficos e imágenes de la interfaz gráfica de usuario.
% 		\end{minipage}}.
% 		.3 images.
% 		.4 panels.
% 		.4 widgets.
% 		.2 modules\DTcomment{ 
% 			\begin{minipage}[t]{8cm}
% 				Las clases que representan a los módulos del Synthi 100.
% 		\end{minipage}}.
% 		.2 SGME\_Settings\DTcomment{ 
% 			\begin{minipage}[t]{8cm}
% 				Archivos de configuración (límites, rangos y otros niveles modificables).
% 		\end{minipage}}.
% 		.2 SynthiGME.sc\DTcomment{ 
% 			\begin{minipage}[t]{8cm}
% 				Clase principal.
% 		\end{minipage}}.
% 		.1 HelpSource\DTcomment{ 
% 			\begin{minipage}[t]{8cm}
% 				Archivos de ayuda y documentación.
% 		\end{minipage}}.
% 		.1 Installation\DTcomment{ 
% 			\begin{minipage}[t]{8cm}
% 				Scripts de ayuda y ejemplos.
% 		\end{minipage}}.
% 		.1 Ladish studio\DTcomment{ 
% 			\begin{minipage}[t]{8cm}
% 				Archivos auxiliares para \emph{Jack} en Linux (prescindibles)
% 		\end{minipage}}.
% 		.1 TouchOSC\DTcomment{ 
% 			\begin{minipage}[t]{8cm}
% 				Layout de ejemplo creado con \emph{TouchOSC}.
% 		\end{minipage}}.
% 		.1 COPYING\DTcomment{ 
% 			\begin{minipage}[t]{8cm}
% 				Archivo con la licencia GNU GPLv3.
% 		\end{minipage}}.
% 		.1 README.md\DTcomment{ 
% 			\begin{minipage}[t]{8cm}
% 				Archivo con la descripción básica de la aplicación.
% 		\end{minipage}}.
% 		.1 SynthiGME.quark\DTcomment{ 
% 			\begin{minipage}[t]{8cm}
% 				Archivo de información del \emph{quark} (para SuperCollider).
% 		\end{minipage}}.
% 	}
%     \source{\propio}
%     \label{fig:arbol_directorios}
% \end{figure}

