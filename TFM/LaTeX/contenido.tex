\begin{document}


  \import{./portada/}{portada.tex}
  \setcounter{page}{2} % Para que tras la portada esté la página 2.

  % \pagecolor{yellow!30}
  % \todototoc
  % \listoftodos[Comentarios y tareas pendientes]
  % \clearpage
  % \nopagecolor

  % Lista de tareas generales TODO
  % \import{./}{todo_general.tex}
  \clearpage

  % Cambio a una página con estilo vacío para la dedicatoria
  \thispagestyle{empty} 
  \import{./capitulos/}{dedicatoria.tex}
  \clearpage
  \thispagestyle{empty} 
  % \import{./capitulos/}{agradecimientos.tex}
  % \thispagestyle{empty} 
  \clearpage
  \thispagestyle{empty} % Asegurar que la siguiente página también esté vacía si la dedicatoria es de una sola página
  
  % Restablecer el estilo de página
  \pagestyle{fancy} 

  \import{./capitulos/}{resumen.tex}
  \clearpage
  

  \phantomsection % Para que el índice apunte al lugar correcto
  \addcontentsline{toc}{chapter}{Índice de contenidos}
  \tableofcontents
  \clearpage

  \phantomsection
  \addcontentsline{toc}{chapter}{Índice de figuras}
  \listoffigures 
  \clearpage

  \phantomsection
  \addcontentsline{toc}{chapter}{Índice de tablas}
  \listoftables
  \clearpage

  \phantomsection
  \addcontentsline{toc}{chapter}{Índice de acrónimos}
  \label{chap:glosario}
  \printglossary[title=Índice de acrónimos, toctitle=Índice de acrónimos]
  \clearpage

  % Inicio del contador de páginas
  \setcounter{inicioContenido}{\value{page}}

  % Importar los capítulos
  \import{./capitulos/}{introduccion.tex}
  \import{./capitulos/}{marco_teorico.tex}
  \import{./capitulos/}{metodologia.tex}
  \import{./capitulos/}{lenguajes.tex}
  \import{./capitulos/}{experimentos_openai.tex}
  \import{./capitulos/}{livecoding_gpt.tex}
  \import{./capitulos/}{composicion_arte_sonoro_GPT4.tex}
  \import{./capitulos/}{resultados.tex}
  \import{./capitulos/}{conclusiones.tex}
  \import{./capitulos/}{limitaciones_prospectiva.tex}

  % página de fin de contenido
  \setcounter{finContenido}{\value{page}}

  \import{./capitulos/}{bibliografia.tex}
  % página de fin de contenido
  \setcounter{antesAnexos}{\value{page}}
  \clearpage


  % Apéndices
  \appendix

  % \begin{titlepage}
  %   \centering
  %   \vspace*{\stretch{1}}
  %   \fontsize{26pt}{20.8pt}\selectfont\textcolor{azul_unir}{Anexos}
  %   \vspace*{\stretch{2}}
  % \end{titlepage}
  % \addcontentsline{toc}{part}{Anexos} % Añade una entrada en el índice
  % \addtocounter{antesAnexos}{2}
  % \setcounter{page}{\value{antesAnexos}}

  % % Restablecer el estilo de página
  % \pagestyle{fancy}

  
  % \import{./anexos/}{system_prompts.tex}
  % \import{./anexos/}{aimuse_manual.tex}
  \import{./anexos/}{algorai_creacion.tex}
  \import{./anexos/}{repositorio.tex}


  % \addcontentsline{toc}{chapter}{Índice de términos}
  % \printindex % Índice de términos. Compilar con makeindex main
  % \clearpage


  \begin{comment}

  \chapter*{Resumen de Páginas}
  \todo{eliminar este contador de páginas}
  % Calcular el número de páginas del contenido principal
  \newcounter{numPaginasContenido}
  \setcounter{numPaginasContenido}{\value{finContenido}}
  \addtocounter{numPaginasContenido}{-\value{inicioContenido}}
  \addtocounter{numPaginasContenido}{1}

  \newcounter{numPaginasRestantes}
  \addtocounter{numPaginasRestantes}{90}
  \addtocounter{numPaginasRestantes}{-\value{numPaginasContenido}}

  Páginas utilizadas: {\Huge\arabic{numPaginasContenido}\ de 90}

  Número de páginas restantes: {\Huge\arabic{numPaginasRestantes}}

\end{comment}

\end{document}