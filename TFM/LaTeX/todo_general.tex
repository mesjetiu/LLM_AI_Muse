
\newpage
\pagecolor{yellow!30}
Importante:
  \todo{Lista de tareas generales para el documento}
  \begin{enumerate}
    \item Revisar todas las expresiones, vocabulario y ortografía.
    \item Uso de comillas españolas consistentemente.
    \item Uso de cursivas en términos en inglés.
    \item Decisión final de términos usados en inglés. Debería ser consistente...
    \item Crear referencias cruzadas entre secciones del texto.
    \item Revisar tamaños de figuras e imágenes, que sea equilibrado.
    \item Seguir revisando textos y abreviaturas desde prompting \ref{sec:llm_tecnicas_prompting}
  \end{enumerate}

  
  Opcional y accesorio:
  \todo{Lista de tareas accesorias para el documento}
  \begin{enumerate}
    \item Comprobar calidad de imágenes creadas por mí en canva.
  \end{enumerate}

% Comando para incluir las páginas del PDF en una línea horizontal


% \begin{tikzpicture}[remember picture,overlay]
%     \foreach \i in {1,...,80} { % Cambia 20 por el número total de páginas de tu PDF
%         \node at (\i*\pageoffset,0) {\includegraphics[page=\i, scale=\scalefactor]{./main_2.pdf}};
%     }
% \end{tikzpicture}

Preguntas para tutoría:
\begin{enumerate}
  \item ¿Puedo variar la jerarquía de capítulos?
  \item ¿Puedo introducir otros como "Consideraciones iniciales"
  \item ¿Dónde añado agradecimientos?
\end{enumerate}
\clearpage
\nopagecolor

