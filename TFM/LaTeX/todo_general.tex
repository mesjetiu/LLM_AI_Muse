Importante:
  \todo{Lista de tareas generales para el documento}
  \begin{enumerate}
    \item Revisar todas las expresiones, vocabulario y ortografía.
    \item Crear referencias cruzadas entre secciones del texto.
    \item Revisar tamaños de figuras e imágenes, que sea equilibrado.
    \item Revisar el formato de todas las citas y referencias bibliográficas (al ser automáticas, pueden faltar datos o estar mal).
    \item Revisar si he puesto consistentemente en source "Elaboración propia" cuando es el caso.
    \item Añadir dedicatoria (en español y checo?) % y huevo de Pascua
    \item Seguir revisando textos y abreviaturas desde prompting \ref{sec:llm_tecnicas_prompting}
  \end{enumerate}

  
  Opcional y accesorio:
  \todo{Lista de tareas accesorias para el documento}
  \begin{enumerate}
    \item Cambiar tipo de letra en códigos y enlaces dentro del texto tipo monospace. \texttt{\textit{Esta es demasiado elegante cuando aparece en cursiva}}. Quizás la solución es no usar cursivas cuando use esta fuente monospace en captions y sources\dots Revisarlo\dots
    \item Añadir icono de audio junto los enlaces a los audios con QR.
    \item Los epígrafes de los capítulos convertirlos en hipervículos a los papers a los que hacen referencia.
    \item Crear un enlace en el que siempre se podrá consultar la última versión del documento.
  \end{enumerate}

% Comando para incluir las páginas del PDF en una línea horizontal


% \begin{tikzpicture}[remember picture,overlay]
%     \foreach \i in {1,...,80} { % Cambia 20 por el número total de páginas de tu PDF
%         \node at (\i*\pageoffset,0) {\includegraphics[page=\i, scale=\scalefactor]{./main_2.pdf}};
%     }
% \end{tikzpicture}
