%!TEX program = lualatex

% Preámbulo

% Documentclass y configuración de página
\documentclass[a4paper,openany,oneside,12pt]{report}
\usepackage[left=3cm,right=2cm,top=2.5cm,bottom=2.5cm]{geometry} % Configuración de márgenes

% Codificación e idioma
%\usepackage[utf8]{inputenc}
%\usepackage[T1]{fontenc}
\usepackage[spanish]{babel}
\renewcommand{\shorthandsspanish}{} 
\addto{\captionsspanish}{\def\chaptername{}} % quita la palabra capítulo de los encabezados.

% Fuentes y espaciado
%\usepackage[sfdefault]{carlito}
%\renewcommand*\rmdefault{carlito}




\usepackage{setspace}
\usepackage{parskip} % Espacio entre párrafos
\setlength{\parskip}{6pt plus 1pt minus 1pt}
\setlength{\parindent}{0pt}
\onehalfspacing

% Otros paquetes
\usepackage{import}
\usepackage[apaciteclassic]{apacite}
\usepackage{fancyhdr}
\usepackage{titlesec}
\usepackage{xcolor}
\usepackage{graphicx} % graficos
\usepackage{comment}
\usepackage{fontspec}

\setmainfont[Path=/usr/share/fonts/TTF/, % Cambiar por la ruta donde se encuentren las fuentes
    BoldItalicFont=calibriz.ttf,
    BoldFont      =calibrib.ttf,
    ItalicFont    =calibrii.ttf,
    FontFace={l}{n}{Font=calibril.ttf},      % Light
    FontFace={l}{it}{Font=calibrili.ttf}    % Light Italic
    ]{calibri.ttf}
\newcommand{\textlight}[1]{{\fontspec{calibril.ttf} #1}}
\newcommand{\textlightit}[1]{{\fontspec{calibrili.ttf} #1}}
    

% Configuraciones
\setcounter{secnumdepth}{3} % indica a Latex contabilizar el nivel de jerarquía de texto hasta el cuarto nivel.
\setcounter{tocdepth}{3}    % indica al comando de generación de tabla de contenido que incluya el cuarto nivel.

% Configuraciones de encabezados y pies de página
\pagestyle{fancy}
\fancyhf{} % Limpia el encabezado y el pie de página
\fancyhead[L]{Carlos Arturo Guerra Parra}
\fancyhead[R]{Título del TFE}
\fancyfoot[C]{\thepage}

% Configuración de las notas al pie
\makeatletter
\renewcommand\@makefntext[1]{%
    \setlength{\parindent}{0pt}%
    \begin{minipage}{\columnwidth}%
    \singlespacing
    \fontsize{10pt}{12pt}\selectfont
    \@makefnmark\ #1
    \end{minipage}}
\makeatother

% Configuración para el título de primer nivel (secciones)
\titleformat{\section}
  {\fontsize{18pt}{21.6pt}\selectfont\color{blue}\bfseries} % tamaño 18pt, interlineado 1.5, azul y negrita
  {\thesection}{1em}{} % Formato del número de sección
  [\vspace{6pt}] % espacio después
\titlespacing*{\section}
  {0pt}{6pt plus 1pt minus 1pt}{0pt} % espacio anterior y después de la sección

% Configuración para el título de segundo nivel (subsecciones)
\titleformat{\subsection}
  {\fontsize{14pt}{21pt}\selectfont\color{blue}\bfseries}
  {\thesubsection}{1em}{}
  [\vspace{6pt}]
\titlespacing*{\subsection}
  {0pt}{6pt plus 1pt minus 1pt}{0pt}

% Configuración para el título de tercer nivel (subsubsecciones)
\titleformat{\subsubsection}
  {\fontsize{12pt}{18pt}\selectfont\color{blue}\bfseries}
  {\thesubsubsection}{1em}{}
  [\vspace{6pt}]
\titlespacing*{\subsubsection}
  {0pt}{6pt plus 1pt minus 1pt}{0pt}

% Contenido del documento
\begin{document}

  \bibliographystyle{apacite}

  \import{./portada/}{portada.tex}
  \import{./capitulos/}{justificacion.tex}

% \chapter{Justificación}
\section{Sección 1}
\subsection{Subsección 1}

Sin duda alguna, se trata de un tema candente que impregna todo, incluso la música.

Modelos de Ia generativa música...

LLM, los más sorprendentes, porque razonan.

Tienen habilidades emergentes, incluyendo la creatividad as artística.

Los LLM se han mostrado muy efectivos en la generación de código de programación, sin excepción de lenguajes orientados a la generación musical. 

Existen trabajos enfocados en la generación de algoritmos en lenguajes de programación, en el que lo que se espera de una función es un resultado matemático, pero aún poco o nada si lo que se espera más allá de la corrección sintáctica es un resultado artístico que apoye las intenciones del creador de sonido o compositor.

Este trabajo se enmarca en un momento vertiginoso, en el que toda investigación es necesaria, al tiempo que está condenada a la obsolescencia en pocos meses. Pero es claro que en un plazo de pocos años no habrá campo intelectual en el que la IA no haya entrado. No se entenderá un profesional de cualquier campo, incluido el artístico, que trabaje al margen de estas herramientas.

No se encontrará aquí ninguna valoración de índole ética o antropológica de la IA, sino más bien una valoración práctica y descriptiva del uso de los LLM en la creación musical en general, y sonora en particular, dando como un hecho que los LLM son capaces de razonar y crear a niveles comparables al humano en el momento de este estudio.

Aunque se ha elegido SuperCollider como el lenguaje objeto del estudio, la investigación es extrapolable a cualquier lenguaje estructurado cuyo destino sea la creación sonora o musical, como Overtone, Pure Data, Sonic Pi... El hecho de que Supercollider es un lenguaje maduro y muy extendido entre la comunidad, al tiempo que razonablemente bien conocido por los LLM actualmente, ha sido uno de los motivos de su elección.

Podria haber sido más atractivo un estudio sobre como los LLM construyen composiciones en un lenguaje más clásico como puede ser el MIDI, si bien, esto claramente dista bastante de lo que estos sistemas son capaces de hacer a día de hoy sin un entrenamiento ad hoc, ya que la complejidad de una composición con notas musicales recae en disciplinas como la armonía, el contrapunto, etc, donde los LLM de propósito general no han sido entrenados lo suficiente a día de hoy.

% Lista de la bibliografía que será mostrada aunque no esté citada.
% \nocite{Csound_book}
% \bibliography{bibtex/biblio}

\end{document}
